\section{Introduction: The Elvira Projects}

This paper presents the Elvira environment, a platform for constructing and using probabilistic graphical models,
which is being built with the support of the Spanish  
{\em Ministerio de Ciencia y Tecnolog\'{\i}a} through two
research projects: Elvira (TIC97-1135-C04, 1997-2000) and Elvira~II
(TIC2001-2973-C05, 2001-2004). The original motivation for the
first project was the existence of several groups in Spain working
in different aspects of probabilistic graphical models  (learning,
propagation, influence diagrams, etc). These groups had important
difficulties in implementing algorithms and maintaining the
existing software. Most of the programming was done in a paper by
paper basis: specific software was created for testing algorithms
proposed in a given moment. This produced a lot of duplications
and, for each group, the available tools were really reduced: some
laboratories could have a great variety of implementations of
learning algorithms, but no  software for propagation. So the main
aim of first Elvira was to integrate in an environment all the
algorithms and procedures developed by the different research
groups. This environment should also have an attractive and easy to
use graphical interface to build and use the models. At first, we
investigated existing tools for Bayesian networks, but we decided
to start from scratch as the primary motivation was to create a
research tool and for that a great deal of flexibility was
necessary that was not provided by any of the examined
environments. Some were commercial oriented with a non open code or
other were very restrictive in their objectives and the work
to adapt them was estimated to be too high. So, we started by
defining a common format to specify and represent Bayesian networks 
(the Elvira format described in section 2) . We chosed a
programming language (Java) and  created a graphical interface 
(see section 3)  in which to integrate the different propagation,
learning, and decision algorithms. Elvira is not a tool for a 
single purpose. So we have not considered only one procedure for
each task. The different possibilities for propagation, learning, and decision algorithms  are described in 
sections 4, 5, and 6 of this paper.

Elvira is not a finished project. In fact, 
Elvira II has just started some months ago. In this project, we
have moved a step forward: apart from improving the
Elvira environment (see section 6 for details), the main objective is now to use this tool to build real applications in several fields: medicine, 
e-commerce, genomics, and agriculture.

Elvira is an open project. It is not restricted to the people working in the laboratories directly involved in the project.
In the web page at address \url{http://leo.ugr.es/~elvira}, the Elvira environment, including the source code, can be freely downloaded.
People is also invited to contribute with new algorithms and ideas.

