\section{The Elvira Format}
In the Elvira program, Bayesian networks and influence Diagrams are
defined by using the {\em Elvira specification format}. This format
is used to maintain both kind of networks in ASCII files (normally
with extension .elv). A network contains three kinds of elements: a
set of nodes, a set of links and a set of relations among the
nodes. A node contains information about one of the variables of
the network. Variables can have a finite number of possible states
(discrete variable) or they can have a continuous domain
(continuous variable). A link specifies the two nodes that it
joins. A link could be directed or undirected. A relation describes
a set of variables connected in some way, and it can provide
numerical information about the variables by means of a {\em
potential}. For example, a relation could represent a conditional
probability distribution or an utility function. Different kinds of
potentials can be used in the  Elvira format, such as tables,
probability trees and functions.

We can define a set of properties for each element of the network
(nodes, links and relations). For example, for nodes we can define
a title, a comment, a position $x$ and $y$ for the graphical
interface, a kind of node (chance, decision or utility). If the
node is a finite-states node, we can define the names of its
states. If the node is continuous we can define the minimum and
maximum values it can take on. For links we can specify if the link
is directed or undirected, and a comment. For relations we can
specify a comment, the kind of relation (potential, conditional
probability), and the numerical information by means of a potential.
Potentials can be represented in different forms. See the
example below as an overview.
The properties of the three kinds of elements take a default value
when they are not specified. A default value can
be redefined as we will show below.

Apart from the properties for each element, there are general 
properties for the network such as
title, comment, author, version, etc. A special kind of general property is the {\em default property}.
A default property defines the value a property will take in the elements of the network, when
it is not specified in that element.

As an example of a Bayesian network in the Elvira format we show the
next network taken from \cite{coo84}

\begin{small}
\begin{verbatim}
bnet Cancer {
  title="Diagnosis of metastatic cancer";
  author="Greg Cooper";
  whochanged="Elvira Consortium";
  default node states=(absent present);
node A { title="Metastatic cancer";
  comment="It is ill or not"; }
node B { title="Serum calcium";
  states=(normal high); }
node C { title="Brain Tumor";
  states = (present absent); }
node D; node E;
link A B; link A C; link B D;
link C D; link C E;
relation A {
  comment = "Probabilities for metastasis";
  kind-of-relation=conditional-prob;
  values= table (0.2 0.8 ); }
relation B A {
  values=table ([high,present]=0.8,
  [high,absent]=0.2,[normal,present]=0.2,
  [normal,absent]=0.8); }
relation C A {
  values= table (0.05 0.2 0.95 0.8 ); }
relation E C {
  values= table (0.8 0.6 0.2 0.4 ); }
relation D B C{
  values=tree (
    case D{
      present=case B{
        high=0.8;
        normal=case C{
          present=0.8; absent=0.05; }
      }
      absent= case B{
        high=0.2;
        normal=case C{
          present=0.2; absent=0.95; }
      }
    }
  );
}
}
\end{verbatim}
\end{small}
