\section{Influence Diagrams}

Influence diagrams (IDs) augment the 
Bayesian network representation with the ability to deal
with events under the decision maker control (decision variables)
and preferences (utility functions). The next two subsections
present the capabilities for
using and  evaluating influence diagrams offered by Elvira. 

\paragraph{IDs representation:}
IDs can be introduced in Elvira with the GUI or through files
with the above specified format. For
IDs we have implemented methods that give information about the
properties of the ID and  prepare the ID for its evaluation,
checking the conditions needed to evaluate it.


\paragraph{IDs evaluation:}
We have implemented two algorithms to evaluate IDs: ArcReversal
and VariableElimination. These methods are not still integrated
in the GUI and must be used from the command line. 

ArcReversal details are presented in \cite{sh86} and VariableElimination 
in \cite{jensen-book2001}. We have
developed various modifications on these algorithms to take advantage of
the use of probability and utility trees and constraints. So Elvira offers 
two variants for these algorithms, adding the capability of using probability 
trees and utility trees for uncertainty and utility potentials.
One of the most frequently described drawbacks of the IDs
is related to its incapability to deal with asymmetric decision problems.
For that reason, we have developed these two new versions to benefit from
qualitative information about constraints (asymmetries)
and to approximate the utility function when its size grows over a certain
limit, using trees for potentials.
