\section{An Ongoing Project: The Future}
In this section, we will enumerate some points that we plan to
develop within the Elvira project in the near future. By means of these improvements,
 our aim is to convert Elvira in an useful tool for the community of
researchers in probabilistic graphical models.

Points to be added to the current Elvira version are related with
the format, the graphical interface, the preprocessing step of data,
inference algorithms and the learning process of a graphical probabilistic structure
from data.

Related with the {\em Elvira format} we aim to develop its
specification on XML format. In this way, we will obtain an
automatic translation to other formats that are currently used by academic and
commercial software in the field of probabilistic graphical
models. We also plan to specify Gaussian variables, as well as
canonical models. While with the former we will model continuous
variables, with canonical models we will avoid the exponential
growth of the probability tables related with the number of parents
of each variable. Finally, we hope to accelerate the reading process
in huge databases.

Desired advances for the {\em graphical interface} are
related with the integration of already implemented learning methods, as
well as the explanation phase. We plan also to add verbal
explanations. We will integrate the already implemented methods to
evaluate influence diagrams, allowing the users to select the
evaluation method and showing the results through the graphical
user interface.


In the {\em preprocessing} step of databases that will
be used by structural learning algorithms, our objective is to implement
filter, wrapper and hybrid methods in order to solve the feature
subset selection problem. We also aim to implement procedures to
transform variables, methods for the imputation of missing data
and algorithms for the discretization of continuous variables.

In relation with {\em inference algorithms}, we plan to add new
algorithms to the wide range of actually implemented ones.
In the category of exact algorithms, we plan to implement the
Shafer--Shenoy scheme, as well as the lazy evaluation for
continuous variables. For both types of methods of evidence
propagation --exact and approximate-- we plan to develop and
implement algorithms for incremental triangulation. The objective
of these algorithms is to obtain a good triangulation for a graph
that is the result of a small modification of an initial graph from
which a good enough triangulation is known in advance. In this
point we will also include some advances in abductive inference. As
the specification of the  explanation set is not an easy task
within the abduction problem, we envision to develop a set of
methods that will try to solve this difficulty. For influence
diagrams we will add MonteCarlo algorithms with the aim to
alleviate the computational burden arised when dealing with complex
decision problems.

In relation with {\em learning} process, our plan is to add some
algorithms to the actually implemented ones, organizing them
with respect to six different characteristics or axes. The first one is related 
with the objective of the learning process. According to it, we
will split the learning algorithms depending whether they try to
show the conditional (in)dependencies between the variables or 
the final objective is to build a supervised or unsupervised classification system.
The second division axis is devoted to the nature of the approach used
to model the data: detection of conditional (in)dependencies or
approaches based on score + search. An hybrid approach can
be also considered. The next characteristic to be taken into account
concerns the complexity of the learned structure: tree, polytree
or multiple connected network. The fourth aspect is related with
the space of search: directed acyclic graphs, the space of
equivalence classes or the space of the orderings between variables.
The fifth axis to split the different learning
algorithms refers to the nature of the variables: discrete, continuous
or mixed. Finally it can be considered whether the
domain to be modeled is static or dynamic.


