\section{Introduction}

Consider a Bayesian network defined for a set of variables $\bX=\{
X_1,\ldots,X_n\}$, where each variable $X_i$ takes values on a
finite set $\Omega_{X_i}$. If $I$ is a set of indices, we will
write $\bX_I$ for the set $\{ X_i|i\in I\}$, and $\Omega_{\bX_I}$
will denote the Cartesian product $\times_{i\in I}\Omega_{X_i}$.
Given $\bx\in \Omega_{\bX_I}$ and $J\subseteq I$,  $\bx_J$ is the
element of $\Omega_{\bX_J}$ obtained from $\bx$ by dropping the
coordinates not in $J$.

\begin{definition} \label{def:approximate} Let $\bE\subset \bX$ be a set
of observed variables and $\be\in\Omega_{\bE}$ the instantiated
value. An algorithm which computes $\hat{p}(x|\be)$, an estimation
of $p(x|\be)$, for each $x\in\Omega_{X}$, $X\in \bX\setminus \bE$
is called an {\em approximate propagation algorithm}.
\end{definition}

Several approximate propagation algorithms have been developed,
which can be classified into two groups: \textit{deterministic}
and \textit{Monte Carlo}.

The more recent deterministic algorithms \cite{Can00,Can02,Can03}
are based on approximating the operations carried out during the
propagation (combination and marginalisation).

On the other hand, Monte Carlo methods consist of transforming the
task of probability propagation into a statistical estimation
problem: a sample is drawn from the Bayesian network and the
posterior probabilities are estimated from it
\cite{Her98,Jen95,Sal2000,Sal01}.

Elvira implements both kinds of algorithms. More precisely, Monte
Carlo algorithms in Elvira are based on importance sampling and
deterministic methods comprise the Penniless propagation scheme
and approximate versions of the Hugin propagation and the variable
elimination algorithm.
