\section{Penniless propagation}

The Penniless propagation algorithm was introduced in
\cite{Can00}, and improved in \cite{Can03}. It is based on
Shenoy-Shafer's, but able to provide results under limited
resources. To achieve this, we will assume that messages are
represented by means of probability trees. The consequence is that
the messages that are sent during the propagation are approximated
by pruning the trees that represent the corresponding
probabilistic potentials.

Another difference with respect to Shenoy-Shafer's algorithm is
the number of stages of the propagation. In the exact algorithm,
there are two stages: firstly, messages are sent from leaves to
the root and secondly, in the opposite direction. The Penniless
algorithm can consist of more than two stages. After the second
phase, the goal is to increase the accuracy of the approximate
messages at each stage, by taking into account the messages coming
from other parts of the join tree. More precisely, when a message
is sent throw an edge, it is approximated conditional on the
message contained in the same edge but in the opposite direction.

The criterion according to which the conditional approximation is
approached in Penniless propagation is to minimise the impact on
the accuracy of the marginal distributions that will be computed
as the result of the propagation process. Taking this into
account, assume we are going to approximate a message, $\phi_{V_i
  \rightarrow V_j}$, between two nodes $V_i$ and $V_j$ in the join
tree, by another potential $\phi'_{V_i \rightarrow V_j}$. After
the propagation, the marginal probability for variables in $V_i
\cap V_j$ will be proportional to $\phi_{V_i \rightarrow V_j}\cdot
\phi_{V_j \rightarrow V_i}$. So $\phi'_{V_i \rightarrow V_j}$
should be computed trying to minimise the value of conditional
information, given by:

\begin{equation}
  \label{eq:distanciacondicional}
  D(\phi_{V_i \rightarrow V_j},\phi'_{V_i \rightarrow V_j}|\phi_{V_j
    \rightarrow V_i}) =
  D(\phi_{V_i \rightarrow V_j}\cdot \phi_{V_j \rightarrow
    V_i},\phi'_{V_i \rightarrow V_j} \cdot \phi_{V_j \rightarrow
    V_i})\enspace .
\end{equation}

The problem is that when a message $\phi_{V_i \rightarrow V_j}$ is
sent, usually the opposite message, $\phi_{V_j \rightarrow V_i}$,
is not exact, but an approximation\footnote{Initially if no
  message has been sent this message is the potential equal to 1 for
  all the configurations} of it, $\phi'_{V_j \rightarrow V_i}$, and
the distance in Equation (\ref{eq:distanciacondicional}) is
conditioned to this approximation. The first time $\phi_{V_j
  \rightarrow V_i}$ is computed,  $\phi'_{V_i \rightarrow V_j}$ was
not available. Once we have it, we can use it to compute a better
approximation of $\phi_{V_j \rightarrow V_i}$ by conditioning on
it and so on. This process can continue until no further change is
achieved in the messages.

If in a given moment some message, say $\phi_{V_i \rightarrow
  V_j}$, is exact, then there is no need to try to improve it by
iterating as described above. Thus, even though the Penniless
algorithm can carry out several iterations, only 2 of them (the
first and the last) are necessary for all the messages, the rest
will be carried out only through approximate messages. It implies
that three different types of stages can be distinguished: the
first stage, the intermediate ones and the last stage.

The goal of the first stage is to collect information from the
entire join tree in the root node, in order to distribute it to
the rest of the graph in a posterior stage. Messages are sent from
leaves to the root, and a flag is kept in every message indicating
whether its computation was exact or approximate.

In the intermediate stages, the approximate messages are updated
according to the information coming from other parts of the tree.
To achieve this, messages are sent in both directions: Firstly,
from the root towards the leaves and secondly, from the leaves
upwards. When a message is going to be sent through an edge, the
flag of the message in the opposite direction is checked; if that
message is labeled as exact, the messages in the subtree
determined by that edge are no further updated. The reason to do
this is that sending messages through that edge will not help to
better approximate the messages in the opposite directions, since
those messages are already exact.

In the last stage the propagation is completed sending messages
from the root to the leaves. In this case, messages are sent
downwards even if the message stored in the opposite direction is
exact. This is done to assure that at the end of the propagation
all the cliques in the join tree have received the corresponding
messages.

The pseudo code for the algorithm is as follows.


\bigskip\noindent
\textsf{\textbf{ALGORITHM PennilessPropagation($\cal
    J$,$stages$)}}

\bigskip\noindent
INPUT: A join tree $\cal J$ and the number of propagation stages.

\noindent
OUTPUT: Join tree $\cal J$ after propagation.

\begin{enumerate}
\item Select a root node $R$.
\item \textsf{\textbf{NavigateUp($R$)}}
\item $stages := stages - 1$
\item WHILE $stages > 2$
  
  \begin{enumerate}
  \item[] \textsf{\textbf{NavigateDownUp($R$)}}
  \item[] $stages := stages - 2$
  \end{enumerate}
  
\item IF ($stages$ == 1)
  
  \begin{itemize}
  \item[] \textsf{\textbf{NavigateDown($R$)}}
  \end{itemize}
  
  ELSE
  
  \begin{itemize}
  \item[] \textsf{\textbf{NavigateDownUpForcingDown($R$)}}
  \end{itemize}
  
\end{enumerate}

The first stage is carried out by calling procedure {\bf NevigateUp}.
This procedure requests a message from each one of its
neighbours, which recursively do the same to all its neighbours downwards
until the leaves are reached. Then, messages are sent upwards
until the root is reached. This task is implemented by the following two
procedures.

\bigskip\noindent
\textsf{\textbf{NavigateUp($R$)}}

\begin{enumerate}
\item FOR each $V\in ne(R)$,
  \begin{itemize}
  \item[] \textsf{\textbf{NavigateUp($R$,$V$)}}
  \end{itemize}
\end{enumerate}


\bigskip\noindent
\textsf{\textbf{NavigateUp($S$,$T$)}}

\begin{enumerate}
\item FOR each $V\in ne(T)-\set{S}$,
  \begin{itemize}
  \item \textsf{\textbf{NavigateUp($T$,$V$)}}
  \end{itemize}
\item \textsf{\textbf{SendApprMessage($T$,$S$)}}
\end{enumerate}


Once the root node has received  messages from all its neighbours, the
intermediate stages begin. This is carried out by the next two
procedures. Observe that exact messages are not updated.

\bigskip\noindent
\textsf{\textbf{NavigateDownUp($R$)}}

\begin{enumerate}
\item FOR each $V\in ne(R)$
  
  \begin{itemize}
  \item[] IF $\phi_{V\rightarrow R}$ is not exact
    \begin{itemize}
    \item[] IF $\phi_{R\rightarrow V}$ is not exact
      \begin{itemize}
      \item[] \textsf{\textbf{SendApprMessage($R$,$V$)}}
      \end{itemize}
    \item[] \textsf{\textbf{NavigateDownUp($R$,$V$)}}
    \end{itemize}
  \end{itemize}
\end{enumerate}


\bigskip\noindent
\textsf{\textbf{NavigateDownUp($S$,$T$)}}

\begin{enumerate}
\item FOR each $V\in ne(T)-\set{S}$
  \begin{itemize}
  \item[] IF $\phi_{V\rightarrow T}$ is not exact
    \begin{itemize}
    \item[] IF $\phi_{T\rightarrow V}$ is not exact
      \begin{itemize}
      \item[] \textsf{\textbf{SendApprMessage($T$,$V$)}}
      \end{itemize}
    \item[] \textsf{\textbf{NavigateDownUp($T$,$V$)}}
    \end{itemize}
  \end{itemize}
  
\item IF $\phi_{T\rightarrow S}$ is not exact
  \begin{itemize}
  \item[] \textsf{\textbf{SendApprMessage($T$,$S$)}}
  \end{itemize}
\end{enumerate}


Finally, the last step is carried out by calling the procedures described
next. After this, the posterior marginal for any variable in the network
can be computed by selecting any node containing that variable and
marginalising its corresponding potential. Observe that after the
intermediate stages, if the total number of stages is odd, we still have
to perform two traversals, one downwards and one upwards. The difference
with respect to the intermediate steps is that in this case, messages are
sent downwards even if they are marked as exact, in order to assure that
the posterior marginals can be obtained in any node.

\bigskip\noindent
\textsf{\textbf{NavigateDownUpForcingDown($R$)}}

\begin{enumerate}
\item FOR each $V\in ne(R)$
  
  \begin{itemize}
  \item[] IF $\phi_{V\rightarrow R}$ is not exact
    \begin{itemize}
    \item[] IF $\phi_{R\rightarrow V}$ is not exact
      \begin{itemize}
      \item[] \textsf{\textbf{SendApprMessage($R$,$V$)}}
      \end{itemize}
    \item[] \textsf{\textbf{NavigateDownUpForcingDown($R$,$V$)}}
    \end{itemize}
    ELSE
    \begin{itemize}
    \item[] IF $\phi_{R\rightarrow V}$ is not exact
      \begin{itemize}
      \item[] \textsf{\textbf{SendApprMessage($R$,$V$)}}
      \end{itemize}
    \item[] \textsf{\textbf{NavigateDown($R$,$V$)}}
    \end{itemize}
  \end{itemize}
\end{enumerate}


\bigskip\noindent
\textsf{\textbf{NavigateDownUpForcingDown($S$,$T$)}}

\begin{enumerate}
\item FOR each $V\in ne(T)-\set{S}$
  
  \begin{itemize}
  \item[] IF $\phi_{V\rightarrow T}$ is not exact
    \begin{itemize}
    \item[] IF $\phi_{T\rightarrow V}$ is not exact
      \begin{itemize}
      \item[] \textsf{\textbf{SendApprMessage($T$,$V$)}}
      \end{itemize}
    \item[] \textsf{\textbf{NavigateDownUpForcingDown($T$,$V$)}}
    \end{itemize}
    ELSE
    \begin{itemize}
    \item[] IF $\phi_{T\rightarrow V}$ is not exact
      \begin{itemize}
      \item[] \textsf{\textbf{SendApprMessage($T$,$V$)}}
      \end{itemize}
    \item[] \textsf{\textbf{NavigateDown($T$,$V$)}}
    \end{itemize}
  \end{itemize}
  
\item \textsf{\textbf{SendApprMessage($T$,$S$)}}
\end{enumerate}

However, if the total number of stages is even, in the final stage we
just have to send messages downwards, which is implemented by the next
procedures.

\bigskip\noindent
\textsf{\textbf{NavigateDown($R$)}}

\begin{enumerate}
\item FOR each $V\in ne(R)$
  \begin{itemize}
  \item[] \textsf{\textbf{SendApprMessage($R$,$V$)}}
  \item[] \textsf{\textbf{NavigateDown($R$,$V$)}}
  \end{itemize}
\end{enumerate}


\bigskip\noindent
\textsf{\textbf{NavigateDown($S$,$T$)}}

\begin{enumerate}
\item FOR each $V\in ne(T)-\set{S}$
  \begin{itemize}
  \item[] \textsf{\textbf{SendApprMessage($T$,$V$)}}
  \item[] \textsf{\textbf{NavigateDown($T$,$V$)}}
  \end{itemize}
\end{enumerate}


Now we show the details of procedure {\bf SendApprMessage}, used in the
algorithms above. A message $\phi_{S\rightarrow T}$ is computed by
combining all the incoming messages of $S$ except that one coming
from $T$ with the potential in $S$, and then, the resulting potential is
approximated according to the message coming from $T$ to
$S$, $\phi_{T\rightarrow S}$.

\bigskip\noindent
\textsf{\textbf{SendApprMessage($S$,$T$)}}

\begin{enumerate}
\item Compute
  $$
  \phi = \left(
    \prod_{V\in ne(S)-\set{T}}\phi_{V\rightarrow S}
  \right)\flechab{S} \enspace .
  $$
\item IF at least one of the messages $\phi_{V\rightarrow S}$ is not
  exact, mark $\phi$ as approximate.
\item Compute $\phi_{S\rightarrow T} = \phi \cdot \phi_S$.
\item IF the size of $\phi_{S\rightarrow T}$ is too big
  \begin{enumerate}
  \item Approximate $\phi_{S\rightarrow T}$ conditional on
    $\phi_{T\rightarrow S}$.
  \item Mark $\phi_{S\rightarrow T}$ as approximate.
  \end{enumerate}
\end{enumerate}


\subsection{Description of the classes related to Penniless
  propagation in Elvira}

The main class that contains the implementation of Penniless
propagation is the class \texttt{Penniless} which is a subclass of
\texttt{propagation}. Here is a description of its main methods:

\begin{itemize}
\item \texttt{Penniless(Bnet b, Evidence e, double[] lp, double[] llp,
    double[] lsp, int[] ls, boolean[] sortAndBound,int m, int triangMethod)}.
  This constructor sets up a propagation object, over which
  the propagation can be carried out.
  \texttt{b} is the network, \texttt{e} the evidence to propagate,
  \texttt{lp} is an array with the limits for pruning in each
  propagation stage (see the section on probability trees),
  \texttt{llp} is an array with
  lower limits for pruning, that mean that, below those
  limits, a pruned tree is considered as exact. \texttt{lsp} is an
  array with limit values per stage under which, leave which
  sum less that the given thresholds are pruned. \texttt{ls} is an
  array indicating the maximum number of leaves per tree in
  each stage, in case that sort and bound is activated.
  \texttt{sortAndBound} is an array indicating whether or not to
  activate sorting and bounding the trees in each stage. \texttt{m} is
  the information measure to use (1 for average and 2 for
  conditional average). \texttt{triangMethod} is the triangulation
  method (0 is for considering evidence during the
  triangulation, 1 is for considering evidence and directly remove relations
  that are conditional distributions when the conditioned variable is removed.
  2 is for not considering evidence.
\item \texttt{public Penniless(Bnet b, Evidence e, double[] lp,
    double[] llp,double[] lsp, int[] ls, boolean[] sortAndBound,
    int m, JoinTree jt)}.
  This constructor is the same as the previous one, but
  now, instead of giving a triangulation method, the join
  tree is provided as argument \texttt{jt}. Some of the
  arguments in these constructors can be set directly for
  already created objects using the next methods:
\item \texttt{public void setLimitForPruning(double[] lp)}.
\item \texttt{public void setLowLimitForPruning(double[] llp)}.
\item \texttt{public void setLimitSumForPruning(double[] lsp)}.
\item \texttt{public void setMaximumSizes(int[] ls)}.
\item \texttt{public void setSortAndBound(boolean[] sAB)}.
\item \texttt{public void setInfoMeasure(int m)}.
\item \texttt{public int getInfoMeasure()}.
\item \texttt{public JoinTree getJoinTree()}. Returns the join
  tree over which the propagation is defined.
\item \texttt{public void setJoinTree(JoinTree jt)}. This
  methods sets the join tree in an object that must be previously
  created.
\item \texttt{public void setKindOfApprPruning(String kind)}.
  Sets the method for pruning the trees, for a propagation object
  previously created. \texttt{kind} can be "AVERAGE" (replace by the
  average), "ZERO" (replace by 0) or "AVERAGEPRODCOND"
  (replace by the conditional average in \cite{Can03}.
\item \texttt{public double obtainEvidenceProbability()}.
  Returns the probability of the evidence contained in the
  propagation object.
\item \texttt{public double obtainEvidenceProbabilityFromRoot()}.
  Gets the probability of the evidence directly from the root,
  as long as a propagation has been carried out previously.
\item \texttt{public void initMessages()}. Initialises all the messages
  to 1, except those corresponding to leaf nodes, which will contain
  the potential in the leaf node. Messages between cliques are marked
  as not exact.
\item \texttt{public void propagate(String resultFile) throws ParseException,
    IOException}. Carries out a propagation over an object
  previously created, and saves the results (the marginal
  distributions) in a text file named \texttt{resultFile}.
\item \texttt{public void sendMessage(NodeJoinTree sender,
    NodeJoinTree recipient, boolean takeNumberNextStage)}. Sends a
  message from a node to another one.
  Marks the messages as not exact when this method carries out
  an approximation or one of the input messages are not exact.
  The message is computed by combining all the messages inwards
  the sender except that one coming from the recipient. Then, the
  result is sorted and bounded conditional to the message
  going from the recipient to the sender.
  It is required that the nodes in the tree be labeled.
  Use method \texttt{setLabels} from \texttt{JoinTree} if
  necessary. The arguments are: \texttt{sender}, the node that sends
  the message. \texttt{recipient}, the node that receives the message.
  \texttt{takeNumberNextStage}. true if we must do
  sort and bound over the message, with
  \texttt{maximumSize[currentStage+1]} and false if we
  must do sort and bound over the message, with
  \texttt{maximumSize[currentStage]}.
\item \texttt{private void navigateDownUp(NodeJoinTree sender)}.
  Send messages from the root (\texttt{sender}) to the leaves, and
  then from the leaves towards the root. The method does not navigate
  throw a branch if the message in the opposite direction
  (\texttt{getOtherValues()}) is exact.
\item \texttt{private void navigateDownUp(NodeJoinTree sender,
    NodeJoinTree recipient)}. Sends messages from the root
  (\texttt{sender}) to the leaves, and then from the leaves
  to the root, through the branch towards node \texttt{recipient}.
  This method do not navigate throw a branch if the message in the opposite
  direction is exact.
\item \texttt{private void navigateDownUpForcingDown(NodeJoinTree
    sender)}. This method is used to send messages from the root
  (\texttt{sender}) to the leaves, and then from the leaves
  to root. When it navigates  down,  if the message in the opposite direction
  in a branch is exact, then it will continue with \texttt{navigateDown}
  over that branch (and not doing the ascending step).
\item \texttt{private void navigateDownUpForcingDown(NodeJoinTree sender,
    NodeJoinTree recipient)}. This method sends messages from the root
  (\texttt{sender}) to the leaves, and then from the leaves
  to the root, through the branch \texttt{recipient}.
  When it navigates  down,  if the message in the opposite direction
  in a branch is exact, then it will
  continue with \texttt{navigateDown} over that branch
  (and not doing the ascending step).
\item \texttt{private void navigateUp(NodeJoinTree sender)}. Sends
  messages from the leaves to the root (\texttt{sender}).
\item \texttt{private void navigateUp(NodeJoinTree sender, NodeJoinTree
    recipient)}. Sends an upward message from \texttt{recipient} to
  \texttt{sender}.
\item \texttt{public void navigateDown(NodeJoinTree sender)}. Sends
  messages from the root (\texttt{sender}) to the leaves.
\item \texttt{public void navigateDown(NodeJoinTree sender, NodeJoinTree
    recipient)}. Sends a downwards message through the branch
  \texttt{recipient}.
\item \texttt{public RelationList getInitialRelations()}. Returns
  the initial relations in the network (the conditional distributions).
\item \texttt{public void computeMarginals()}. Computes the marginals
  after a propagation and put them into the instance variable
  \texttt{results}. Sets an entry in the hash table \texttt{positions}
  for each variable, to indicate the index to locate the variable
  in vector \texttt{results}.
  
  \subsection{Running Penniless propagation}
  
  In order to run Penniless propagation from the command line, go
  to the Elvira root directory and type
  \verb"'java elvira.inference.clustering.Penniless NetworkFile"
  \verb"OutputFile StatisticsFile ExactResultsFile"
  \verb"kindOfApprPruning(AVERAGE|ZERO|AVERAGEPRODCOND) infoMeasure(1|2) NumberOfStages LimitForPruningStage1"
  \verb"LimitForPruningStage2 ... LowLimitForPruningStage1 LowLimitForPruningStage2 ... LimitSumForPruning1"
  \verb"LimitSumForPruning2 ... MaxSizeInStage1 MaxSizeInStage2 ... SortAndBoundInStage1(true|false)"
  \verb"SortAndBoundInStage2 ... TriangulationMethod(0|1|2) [EvidenceFile]'". Argument \verb"ExactResultsFile" can be
  replaced by the word \verb"NORESULTS", indicating that no
  exact results are available, in which case the error is not
  computed.
  
\end{itemize}
