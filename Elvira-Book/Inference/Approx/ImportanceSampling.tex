\section{Importance Sampling}

Importance sampling is well known as a variance reduction
technique for estimating integrals by means of Monte Carlo methods
(see, for instance, \cite{Rub81}). Here we study how to use it to
estimate conditional probabilities.

The technique is based on sampling configurations $\bx \in
\Omega_{\bX}$ with a given probability function $p^*:\Omega_{\bX}
\rightarrow [0,1]$, verifying that $p^*(\bx)>0$ for every point
$\bx$ such that $p(\bx,\be )>0$. Then for each configuration, its
weight is computed:
\begin{equation}
  w(\bx) = \frac{p(\bx,\be)}{p^*(\bx)} \enspace .
\end{equation}

If we obtain a sample of size $m$ $(\bx^1,\ldots,\bx^m)$ with
its corresponding weights $(w(\bx^1),\ldots,w(\bx^m))$ we have the
following facts:

\begin{enumerate}
\item The average of the weights $\sum_{i=1}^{m}w(\bx^i)$ is an
  unbiased estimator $\hat{p}(\be)$ of the probability of the
  evidence $p(\be)$.
  
\item If $p^*(\bx) \propto p(\bx,\be)$ then the variance of this
  estimator is 0. In general, this is not possible, but we should
  select $p^*(\bx)$ as close as possible to $p(\bx,\be)$.
  
\item For any event $A$ (for example, the event $A \equiv
  [X_i=x_i]$) the weighted average of the cases in the sample in
  which the event $A$ is verified is an unbiased estimator
  $\hat{p}(A|\be)$ of $p(A|\be)$:
  \begin{equation}
    \hat{p}(A|\be) = \frac{\sum_{A \mbox{ is verified in } \bx^j}
      { w(\bx^j) }}{ \sum_{j=1}^{m}w(\bx^j)}
    \enspace .
    \label{eq:estimation}
  \end{equation}
  
\item If $p^*(\bx) \propto p(\bx,\be)$, then the variance of this
  estimator is $p(A|\be)(1-p(A|\be))/m$.

\end{enumerate}

The general structure of an importance sampling algorithm is as follows:

\bigskip\noindent
\textsf{\textbf{Importance\_Sampling}}

\begin{enumerate}
  
\item For $j:=1$ to $m$ (sample size)
  \begin{enumerate}
  \item $T:=\{ p_i^{R(\bE=\be)}, ~ i=1,\ldots,n\} \cup \{ \delta_X | X\in\bE\}$
  \item Generate a configuration $\bx^{(j)}$ using $p^*$.
    Calculate
    \begin{equation}
      w_j:=\frac{\left( \prod_{f\in
            T}f(\bx^{(j)}_f)\right)}{p^*(\bx^{(j)}) } \enspace .
      \label{peso}
    \end{equation}
  \end{enumerate}
\item For each $x_i \in \Omega_{X_i}$, estimate $p(x_i , \be)$
  using equation (\ref{eq:estimation}) where $A$ represents the event
  $[X_i=x_i]$.
\item Normalise values $p(x_i , \be)$
  in order to obtain $p(x_i | \be)$.
\end{enumerate}

In the above algorithm $\bx^{(j)}_f$ denotes the configuration $\bx^{(j)}$
obtained by leaving only the coordinates of the variables for which
$f$ is defined.

The determinant aspects of an importance sampling  algorithm are the
computation of the sampling distribution $p^*$ and the sampling
procedure. Regarding the obtainment of the sampling distribution, the
next methods are considered in Elvira:

\begin{itemize}
\item Use the conditional distribution to simulate the variables in
  topological order. This is known as {\em forward sampling} or {\em
    probabilistic logic sampling}.
  \cite{Hen88}.
\item Obtain the sampling distribution through a pre-computation
  consisting of an approximate variable elimination propagation,
  either using probability tables \cite{Her98} or probability trees
  \cite{Sal2000}.
\end{itemize}

Attending to the sampling procedure, there are two versions of
importance sampling implemented in Elvira. One of them is based on the
use of antithetic variables \cite{Sal01} and the other one relies on
the dynamic update of the sampling distribution according to the
configurations previously simulated \cite{Mor03}.