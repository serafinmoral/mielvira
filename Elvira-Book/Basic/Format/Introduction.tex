\section{Introduction}

Aqu� empezamos con el listado y ahora intentamos hacer referencia
\ref{Listado}





\begin{lstlisting}[frame=trBL, caption=Relacion Can�nica, label=Listado]{}
/* Bnet.java */

package elvira;


import java.util.*;
import java.awt.*;
import java.io.*;
import java.net.URL;
import elvira.inference.*;
import elvira.inference.clustering.*;
import elvira.inference.elimination.*;
import elvira.inference.approximate.*;
import elvira.inference.abduction.*;
import elvira.parser.*;
import elvira.potential.*;
import elvira.gui.KmpesDialog;

/**
 * This class implements the structure for storing and
 * manipulating the Bayesian Networks.
 *
 * @version 0.1
 * @since 20/11/2000
 */


public class Bnet extends Network {


/**
 * Frequently used values.
 */
public static final String ABSENT = "Absent";
public static final String PRESENT = "Present";

/**
 * <code>true</code> if the network has been compiled or <code>false</code>
 * otherwise.
 */
private boolean isCompiled = false;


/**
 * Program to check the performance from the command line.
 */

public static void main(String args[]) throws elvira.parser.ParseException, IOException {

  Bnet b;
  int nparents, nnodes, ncases;
  Random generator;
  FileWriter f;

  if (args.length < 4) {
    System.out.print("Too few arguments. Arguments are: file,number of nodes,number of cases,number of parents");
  }
  else {
    f = new FileWriter(args[0]);
    nnodes = (Integer.valueOf(args[1])).intValue();
    ncases = (Integer.valueOf(args[2])).intValue();
    nparents = (Integer.valueOf(args[3])).intValue();
    generator = new Random();
    b = new Bnet(generator,nnodes,nparents,ncases,true,1,0.009);
    b.setName(args[0].substring(0,args[0].length()-4));
    b.saveBnet(f);
    f.close();
  }
}

\end{lstlisting}


\cite{Can00,jensen-book2001}
