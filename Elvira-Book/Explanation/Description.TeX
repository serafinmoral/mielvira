\section{Description of the explanation method through the GUI}
\label{sec:description} In the case of Bayesian networks, there
are three basic issues that can (must) be explained, as we have
said before: the knowledge base, the reasoning process performed
by the system to obtain (or not) a conclusion and the evidence
propagated, if any. As it has been previously said, Elvira has two
modes: \emph{Edit}, for editing the graphical model, and
\emph{Inference}, for processing it and in which most of the
explanation capabilities are provided.

\subsection{Explanation of evidence}

It consists in determining which values of the unobserved
variables justify the available evidence. This process is usually
called \emph{abduction}, and is based on the (usually implicit)
assumption that there is a causal model. Since this topic has been
explained with more detail in chapter 6, we will only remark that
the purpose of this kind of explanation is basically to offer a
diagnosis for a set of observed anomalies. For instance, in
medical expert systems, an explanation consists of determining the
disease or diseases that explain the evidence: symptoms, signs,
test results, etc.


\subsection{Explanation of the model} \label{sec:static}Also known as \textit{static explanation}
\cite{Henrion90}, it consists in displaying the information
contained in the knowledge base: nodes, links and the whole
network. The objective is either to permit a human expert to
examine the content of the knowledge base during the phase of
\emph{construction} of the expert system or to offer a novice user
some knowledge about the domain for \emph{instructional purposes}.

\begin{description}
\item [Explanation of nodes] The verbal explanation of a selected node
is displayed on a new window which appears when the \emph{Explain
Node} option in the contextual menu of the node is selected. It
contains the following information: name, states, parents and
children, prior odds, posterior odds, purpose and importance
factor. The \emph{purpose} of a node is defined by the role played
by such a node in the network. In fact one of the differences
between Elvira and other tools is that nodes are classified into
several categories, such as \textit{symptom} or \textit{disease}
or some other roles defined by the user according to his/her
needs. The \emph{importance factor} is a value assigned by the
human expert and represents the importance of that node with
respect to the rest of the nodes. Elvira also allows the user to
navigate across the network. For the graphical explanation, nodes
can be \emph{expanded}, through the \emph{Expand Node} option in
the contextual menu or in the \emph{Inference} menu, or by
clicking on the third icon in the tool bar. Expanded nodes are
drawn with their name, their states, the graphical representation
of their associated probabilities and their numerical value, whose
precision can be selected by way of the \emph{Precision} option of
the \emph{View} menu. When nodes are not expanded they are
\emph{contracted}, in whose case they are represented by an oval
containing only its name or its title, depending the option the
user had selected in the \emph{View} menu.
\item [Explanation of links.] ELvira can display the
links of the network in different colors in order to offer
qualitative insight about the conditional probability tables
\cite{lacave03e}, based on the work about qualitative influences
made previously by \cite{Wellman90a}. This option is offered both
in Edit and Inference mode by selecting the \emph{Influences}
option of the \emph{View }menu or by clicking on the icon with the
red arrow crossed by a blue curve in the corresponding tool bar.
Moreover, it is also possible to offer verbal explanations of a
selected link trough the \emph{Explain Link} option of its
contextual menu. In this case, a new window appears which contains
the origin and destination of the link, the kind of relation it
represents (IS-A, HAS-A, etc.) and the likelihood ratios for each
state of the destination node. This option allows the user to
understand the states of the origin which better explain each
state of the destination.
\item [Explanation of the network]
The classification of nodes is used for generating verbal
explanations of the whole network, which are displayed if the user
selects the \emph{Explain Network} option when the contextual menu
is open at any point of the inference panel and no node or link is
being selected. It is also capable of offering \textbf{graphical}
explanations, by explaining all nodes and links graphically, as
indicated above. However, if the network has many nodes, the
graphical display of all this information can be misleading for
the user. In this case, he/she can simplify the explanation,
selecting the nodes in the network to expand, by combining both
the role and the importance factor of the nodes in the tool bar.
This allows the user to display graphically only the nodes whose
importance factor is greater than an \emph{importance threshold}
and have a given role.
\end{description}

\subsection{Explanation of the reasoning} \label{sec:dynamic}This is sometimes
known as \textit{dynamic explanation} \cite{Henrion90} and it
tries to justify the results obtained by the system and the
reasoning process that produced them. The most known methods for
generating explanations in Bayesian networks \cite{lacave02} are
related with \emph{sensitivity analysis to the parameters}
\cite{Castillo97, Chan02, Coupe00, Laskey95}, in order to analyse
how sensitive are the conditional probabilities to the parameters,
or \emph{sensitivity analysis to the evidence,}
\cite{jensen-book2001, Suermondt92} which consists in studying the
effect the evidence exerts on the conditional probabilities of one
or more variables.

We have implemented in Elvira the following options for explaining
the reasoning performed on a Bayesian network:
\begin{description}
  \item [Graphical display of probabilities] in the expanded nodes of the network, because it is one of the most
  intuitive ways to understand the results.
  \item [Sensitivity analysis to the evidence] by managing several evidence cases simultaneously and classifying the
findings depending on the impact they have over a variable. These
features of Elvira distinguishes it from the rest of software
tools and they are based on the concept of \emph{Evidence Case},
which is defined by the set of findings of the evidence and the
probabilities associated to the unobserved nodes. The \emph{prior
case} corresponds to the absence of evidence and its associated
probabilities are the prior probabilities of every node in the
network. To quickly identify observed nodes from the rest, they
are coloured in gray.

Taken this into account, the method provides the user different
tools for analysing the effects of all the evidence has over the
whole network or over one variable, as well as for studying the
impact that each finding exerts individually, as we describe next:
\begin{description}
  \item \textbf{SA over the whole network} The idea consists in
  displaying graphically each node of the network in different colours according
   to the \underline{kind} and \underline{amount} of influence received by
   the evidence \cite{lacave03e}, inspired also in Wellman's work on
\emph{qualitative probabilistic networks} \cite{Wellman90a}. When
the user selects the \emph{Automatic} option in the
\emph{Explanation Options} of the Options menu, nodes receiving a
positive influence from the evidence of the selected evidence case
are coloured in red; colour blue is chosen for the negative
influences and purple for the unknown influences.

  \item \textbf{SA over one variable}

  In case the user wants to focus only on certain variable, the user has to click on the icon
 represented by a question tag in the tool bar and a new windows appears where the required variable must be selected.
 Then, a table is displayed containing the name of each of its states together
with their prior probabilities, the posterior probabilities and
the logarithmic ratio of both probabilities. Moreover, he/she can
get a \textbf{analysis of the findings} of the evidence which have
a positive or negative influence over such variable, in two ways:
\begin{itemize}
  \item \emph{Automatically,} showing textually the kind and amount of influence each
  finding has over one variable after pushing over the \emph{Why?} button in that window.

  \item \emph{On demand,} when evidence has not too many findings
  and the network is not too large. In this situation, the user
  can create one case with all the evidence and others cases excluding
  each finding of the evidence.
  \end{itemize}

  Apart from this, Elvira can \textbf{display of the chains of reasoning}
  through which evidence flows if the user pushes over the \emph{How?} button in that window.
  This is a very useful tool because it allows the user to focus only on the portion of the network
  affected by the evidence. The algorithm used for it is the one proposed
  by Suermondt \cite{Suermondt92}. The idea is to calculate all the predecessors of the nodes of the evidence and of the selected
  variable $h$. After obtaining their related nodes, then nodes $n$ which are d-separated from $h$ by
  the evidence except $n$ are removed. Finally, the nodes included in all paths from evidence to $h$ which contains only related nodes and which are not
  d-separated are marked. Moreover, we propose the colouring of
  the nodes in these paths depending on the kind and amount of influence
  received by the evidence. The whole network can be displayed again by clicking on the last icon in the tool bar.
\end{description}
\end{description}

To conclude this section, we describe below some interesting
facilities provided by Elvira to help the user to manage the
different evidence cases.

\subsection{Management of the evidence cases}
\label{sec:caseslist} At each moment, there can be several
evidence cases stored and one of them is the \emph{current}
evidence case. A different bar is displayed for each case in the
expanded nodes, although only the numerical probability of the
current case is displayed. The colour of the bars and of the
highlighting rectangles are specific of each case. New findings
are added to the current evidence case and the background colour
of the window changes to gray, so that the user can easily
identify the findings of the current case. All the findings
introduced by the user are added to this case, until he decides to
generate a new case or to select one of the cases previously
introduced. Of course, the user can also remove some of the
findings of the current case. In the case of the prior case, since
it does not admit any evidence, when switching to inference
mode, a second case is automatically generated, in order to store the findings entered by the user. \\
The first two icons of explanation bar are respectively for saving
an evidence case to a file and for generating a new case. There is
also a text field that displays the name and colour of the current
case, together with buttons for navigating across the set of
evidence cases. The following two icons besides those serves for
deleting the evidence of the current evidence case and for
removing every evidence case from the list of cases but the prior
case.
\begin{itemize}
\item  {Monitor of cases}. If the user also wishes to
control which cases are stored and displayed, he/she can open the
monitor of cases by selecting this option in the \emph{Inference}
menu or clicking on the corresponding icon in the tool bar. This
monitor contains two icons for creating and editing evidence
cases, a \emph{Delete} button for removing cases from memory, and
the \emph{Options} button for setting the number of
stored/displayed cases and other properties. Each stored case
corresponds to a row in a three-column table: the first cell of
the row contains a checkbox for indicating whether the stored case
is displayed or not; the second cell is an editable field for
introducing the name of the case; and the third cell allows the
user to change the color assigned to that case.

\item  {Editor of Cases}.In Elvira, a finding can be introduced by double-clicking on the
corresponding line of the expanded node or by opening the Editor
of Cases after clicking on its icon in the tool bar or by
selecting that option in the \emph{Inference} menu. A pull-down
menu gives access to the list of variables in the network, so that
the user can enter additional findings. The main utility of this
editor is to enter or remove evidence when the number of variables
is so high that it becomes impossible or cumbersome to display the
whole network on a screen.\end{itemize}
