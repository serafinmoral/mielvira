\section{Implementation}

As it has already been said, {\em Elvira} is implemented in Java
and its source code is organized in several packages, focusing
each one in different tasks. The one for the storage of the
classes for generating explanations is the package {\em
elvira.gui.explication}, included in the package {\em elvira.gui}
because every explanation capability is offered through the GUI.
Therefore, in this section I will focus in the classes of the
package {\em elvira.gui.explication}, as well as I will present
very briefly certain classes of the package {\em elvira.gui} which
are also used to generate explanations. I will finish this section
describing the package {\em elvira.localize} which contains the
classes needed for the internationalization of the Elvira GUI.

\subsection{Basics of the GUI: Package \emph{elvira.gui}}

The main class of the GUI is \emph{ElviraFrame} because it
contains all the items of the menus, all the toolbar buttons and
the listeners for all these items and buttons. In fact, the
invocation to the \emph{Elvira} program trough the GUI implies the
creation of an instance of \emph{ElviraFrame} which controls every
task during the performance of the program. This class also
contains the necessary intermediate structures and variables to
manage all the internal frames of Elvira, among which the most
important are, basically, the frame for displaying the messages,
implemented by the class \emph{MessageFrame}, and the different
frames associated to the different models which can be opened at
the same time, each one defined by an instance of the class
\emph{NetworkFrame}, and all of them grouped into an
\emph{ElviraDesktopPane}. Although several networks or influence
diagrams can be opened at the same time, only the
\emph{currentNetworkFrame} is displayed at a time.

A \emph{networkFrame} contains the elements needed for managing
the different aspects of the windows depending on the working mode
(edit or inference) or if the model is going to be learnt from a
data base. The most important methods of NetworkFrame are:
\begin{itemize}
  \item public void activeEditorPanel(), which only changes to a panel for
editing the model and set to false the attribute
\emph{isCompiled};
  \item public void activeInferencePanel(), which can compile the
model for getting the prior probabilities, using the selected
method by the user, only if the mode \emph{autopropagation} has
been chosen in the options for doing inference. Moreover, the list
of cases for managing several evidence cases simultaneously is
created (see Sec. \ref{sec:caseslist}). If the model has been
compiled, the list contains as its first element the \emph{prior
case}; if not, the list remains empty;
  \item public void activeLearningPanel(), which initializes the elements needed to learn a Bayesian network from a data
base.
\end{itemize}

This is because since each of the three different tasks to be done
in Elvira (edit, inference or learning) is performed under
different views, we have defined the classes \emph{EditorPanel,
InferencePanel} and \emph{LearningPanel} representing,
respectively, the three different panels associated to each task,
all of them inheriting of class \emph{ElviraPanel}. This class
contains the constants representing the colors, fonts, size, etc.
needed for display a graphical model, the elements selected (if
any) by the user at any moment, stored as an instance of class
\emph{Selection}, the
elements for managing the zoom, etc.\\

The class \emph{EditorPanel} represents the work area for managing
every task related with the graphical edition of a model by
clicking and dragging the mouse, which implies basically the
addition or removing of nodes and links and the definition of
their properties. With such a purpose, some methods are overrided:
\begin{itemize}
\item public void mouseDragged(MouseEvent evt), to calculate the
new position of the nodes and the links displayed in the
editorPanel in order to repaint it, if needed;
\item public void mousePressed(MouseEvent evt), for creating a node
or a link, or for selecting one or several nodes or links. There
is a class, called \emph{Selection}, to store the items selected
by the user.
\item public void mouseReleased(MouseEvent evt), to update the position of
the selected nodes and links when they are moved, or to create a
new arc when the mouse is released on a node;
\end{itemize}
Another important method of this class is \emph{public void
paint(Graphics g)}, which displays the model in the panel. Since
Elvira provides the user the possibility of visualizing the
graphical representations of the influences transmitted by the
links, this method needs to make some callings to certain other
methods of the class \emph{macroexplanation}, which will be
described in section \ref{sec:impexplanation}.

Moreover, those methods contain callings to the constructors of
the classes in charged of displaying the windows for helping the
user to edit the properties of a selected node or link or even the
whole network, which are, respectively:
\begin{itemize}
  \item \emph{EditVariableDialog}, which implements a panel with for tabs, each one for
  defining the properties, states, parents and relation of a node. A relation can be defined by a canonical model,
  implemented by class \emph{CanonicalPotential}.
  \item \emph{LinkPropertiesDialog}, which contains the tail and
  the head of the link and if it is directed.
  \item \emph{NetworkPropertiesDialog}, for setting the properties of the model, such as its title, author, comments, etc.
\end{itemize}


The class \emph{InferencePanel} is defined to manage the
propagation of evidence but also to provide every explanation
capabilities described in Section \ref{sec:description}. Since
this topic is more deepen described in the next section, we will
focus here only in the main aspects related to the introduction of
evidence and its propagation. In this sense, this class contains
one important method, \emph{public boolean propagate(Case c)} for
propagating an evidence case c, and two interesting attributes:
\begin{itemize}
  \item int INFERENCEAIM, to know if the user wants to do evidence
  propagation or abduction;
  \item int inferenceMethod, to store the method used for such
  task;
\end{itemize}
The values of these attributes are set by the user thanks to the
window implemented by the class \emph{PropagationDialog}.\\

With respect to the introduction of evidence, it is made by simply
clicking twice on the node which is observed and then controlled
in the class by the method \emph{public void
mousePressed(MouseEvent evt)}. If the node is
\underline{contracted}, i.e., drawn only by an oval containing its
name, a new window appears providing a list of the states of the
node for setting the finding. This window is
defined by the class \emph{FindingDialog}.\\

Finally, since all aspects related to class \emph{LearningPanel}
have been described in chapters 9 and 10 we won't repeat it again
here.

\subsection{Explanation capabilities: Package \emph{elvira.gui.explication}}
\label{sec:impexplanation}

This package contains most of the classes needed to implement the
explanation capabilities described in Section
\ref{sec:description}. We will first describe how we have
implemented the options for the explanation of the model and then,
for the reasoning, focusing on the case of discrete variables,
although naming the corresponding classes for the continuous case.


\subsubsection{Static explanation}

For generating \textbf{verbal} explanations of the model (see Sec.
\ref{sec:static}) we have added in class \emph{Node} two
attributes: \emph{private double relevance}, to define the
subjective importance of nodes; and \emph{private String purpose},
to represent the rol of each node in the network. Moreover, we
have implemented three classes:\begin{itemize}
\item The
verbal explanation of the nodes are generated by way of the class
\emph{ExplainNodeDialog} which defines a dialog for presenting the
explanation and the main methods are:
\begin{itemize} \item public String razonprob(double[] pr), which
generates a String which contains the verbal explanation of the
ratio of the probabilities stored in the array \emph{p}. If the
node is binary, there is only one ratio; in other case, the states
are ``ordered'' according to their probabilities from the biggest
to the lowest and then the ratios are presenting following this
order;
\item public void setOrPreNode(), which stores in the array \emph{priori} the
prior probabilities of each state of the node;
\item public void setOrPostNode(), similar to \emph{setOrPreNode}
but it stores the probabilities after propagating certain
evidence.
\end{itemize}

\item In a similar way, the class \emph{ExplainLinkDialog} defines a
dialog for presenting the information of a selected link, being
its most important methods the following:
\begin{itemize}
  \item public void setlikelihood(), which stores in an array the
  likelihoods provided by the link. If the head of the link has
  only one parent, those are computed taking into account the
  probability tables; in other cases, the tail is instantiated in
  order to propagate its values and obtain the likelihoods.
  \item public String razonprob(double[] pr, FiniteStates
  current), similar to the same method in
  \emph{ExplainNodeDialog}.
\end{itemize}

\item The class \emph{ExplainNetDialog} implements a dialog for
presenting the verbal explanation of a net, taking into account
the rol of the nodes. With such a purpose, it defines a String for
each of the predefined rol (\emph{causesdis, effectsdis}, etc.),
in order to generate the information associated to each disease.
As well it includes the following methods:
\begin{itemize}
  \item void fillEditor(), which selects the nodes whose purpose
  is ``Disease'' in order to classify its ascendants and descendants and then generate the corresponding text;
  \item void classifynodes(Vector asc, boolean up), which takes
  the vector which contains the ascendants (if up is true) or the
  descendants (if up is false) of the disease and classify them in
  several Strings (causesdis, effectsdis, etc.) depending on their purpose;
  \item String filltext(String nodename), which takes those Strings and generates the final
  text according to the elements stored in them.
\end{itemize}

\end{itemize}

For the \textbf{graphical} explanations, we have implemented
several classes for representing the graphical properties of nodes
and links:
\begin{itemize}
  \item The class \emph{VisualNode} contains as constants the weight and height needed for draw a node (chance, decision or utility),
  either contracted or expanded. For the case of expanded nodes, there are other classes
  for defining some properties, taking into account if the nodes are
  discrete, for which we have implemented the class \emph{ExplanationFStates}, or continuous,
  in whose case the classes \emph{ExplanationContinuous} and
  \emph{ExplanationDensity} have been defined in a similar way.
  The main methods of the class VisualNode are:
\begin{itemize}
  \item public static void drawNode(Node node, Graphics2D g, Color nodeColor,
        Color nodeNameColor, boolean byTitle, boolean dragged), which draws a node with a width according to
the length of its name and its kind, taking into account the color
of the node, the color of the string contained into the name, if
the string displayed into the node will be the title of the node
and if the node must be drawn in the dragged mode.
  \item public static void drawExpandedNode (Node node,Graphics g, Color nodeColor,
                  Color nodeNameColor, Node n,
                  boolean byTitle, boolean dragged,
                  FontMetrics fm), similar to the previous one but which draws the node when the inference mode is
                  enabled and the user wants to see the probability distributions into the net.
                  It uses a Node n, as a ExplanationFStates or ExplanationContinuous  variable
 to draw the propagation result.
\end{itemize}

  \item The class \emph{ExplanationFStates} inherits from \emph{FiniteStates} and
  contains as attributes the node to be expanded, the list of cases to be graphically
  displayed in each expanded node, the names of the states of the node, the prior probabilities
  of each one and the posterior probabilities of each state of the node after propagating some
  evidence. Each set of probabilities is stored in a matrix. Among its most important methods are:
\begin{itemize}
  \item private void setPriori(), which saves the prior probabilities in the \emph{priorprob}
  array, searching them among the list of potentials (\emph{bnet.getCompiledPotentialList()}) obtained when compiling the model;
  \item public int getMaxProbState(double[] prob), which returns
  the position of the matrix which contains the maximum value. If there is more than one maximum, returns
  -1.
\end{itemize}
  \item The class \emph{VisualExplanationFStates} defines the
  panel in which the expanded node is drawn. Its constructor needs as parameter an instance of ExplanationFStates.
  Moreover, in order to display all the information associated to each state, it defines the
  attribute \emph{visualfstates} which is an array of
  \emph{VisualFStates}. Their main methods are:
\begin{itemize}
  \item public void paintExplanation(Graphics2D g, int posx, int posy),
  which draw the names of the states and the graphical
  representation of their probabilities by invoking to the
  following methods:
  \item public void paintStatesNames(Graphics2D g, int posx, int
  posy), which paints the names of the states from position (posx,posy), marking the most
  probable;
 \item public void paintpriori(Graphics g, int posx, int posy) and public void paintsaved (Graphics2D g, int posx, int posy), which
 create the bars corresponding to the graphical representation of the (prior and saved) probabilities of each state
 (defined by the class \emph{VisualFStatesDistribution}) and paint them.
\end{itemize}
  \item The class \emph{VisualFStates} is a Vector whose first element corresponds to the graphical representation of the
  prior probability of that state, and the rest correspond to the different probabilities obtained after propagating the evidence cases created by the
  user. In the expanded nodes, the probabilities of a node are graphically represented as diagram
  bars, each one of them defined by an instance of the class
  \emph{VisualFStatesDistribution}.
  \item The class \emph{VisualFStatesDistribution} is implemented
  in order to define the graphical representation of a probability
  value. It has as constants the \emph{weight} of the bar and the \emph{height} of the area in which it is going to be drawn.
  Other main attributes are the ones for defining the coordinates
  \emph{posx} and \emph{posy} of the bar, its \emph{colour} and if it is going
  to be \emph{visible} or not. Its most important method is \emph{public void paintFSD(Graphics g, int posx, int posy, boolean
  b)}, which
  \item The class \emph{VisualLink} contains among its constants the
  size and width of the arcs. Its most important method is \emph{public static void drawArc(Link link, Graphics2D g, boolean dragging,
  Selection  s)}, which draws an arc taking into account the size of the head node in order to calculate the position of the arrow.
  If the link is being dragged it must be drawn as a dashed line. If the link is not selected but the node that
contains it is, the link must be drawn as if it were selected. The
method uses the parameter \emph{s} to know if the link must appear
when it is dragged dashed.
\end{itemize}

Moreover, we have included several attributes and methods in class
\emph{InferencePanel}:
\begin{itemize}
  \item private double expansionThreshold, which defines the threshold
  upon which nodes are going to be expanded when changing from
  edit to inference mode;
  \item public int expandMode, for defining the policy to be applied when
  the Expand/Contract button is clicked;
  \item private boolean purposeMode, which indicates if the
  purpose of the nodes is going to be used for expanding the
  nodes. They are selected thanks to the window defined by the class \emph{PurposeDialog}, which after selecting the
  roles it invokes to the method \emph{setFunctionThreshold} of InferencePanel;
  \item private boolean showInfluences, to know if the
  influences of the links are going to be graphically represented;
  \item public void expandNodes(double threshold, String[]
  functions), which expand the nodes whose relevance is greater
  than \emph{threshold} and whose purpose is included in the \emph{functions}
  matrix.
  \item public void paint(Graphics g), which draws the network
  taking into account the values of those attributes. If
  showInfluences is true, there are callings to some methods
  of class \emph{macroExplanation}, which is described above.
\end{itemize}

The class \emph{macroExplanation} has been implemented in order to
define some static methods for providing some explanation tools
and which could be used by different classes. In particular, it
contains, among others, the methods needed for representing the
graphical influences transmitted by the links, which are the
following:
\begin{itemize}
  \item public static double[][][] greaterdist(Bnet bnet, Node n, Node
  m), which returns in a three dimensional matrix the cumulative
  distribution of a node \emph{n}, given one of its parents \emph{m} and the \emph{bnet}
  in which they are included, i.e. it calculates $P(n\geq n_{i}|m) \mbox{ for all state }n_{i}\mbox{ of
  }n$. The first dimension of the matrix is defined by the states of
  $n$. The second one is defined by the total number of
  distributions to be computed which is equal to the product of
  the states of the parents of $n$ except $m$. The third dimension is defined
  by the states of $m$.
   \item public static int compare(double[][][] dist), which given
   a matrix with the cumulative distribution as described above,
   returns 0 if the rows are ordered, being the last one the
   greatest; returns 1 if the order is the inverse; returns 2 if
   all of them are equal and 3 if an order can not be established between the
   rows. These two methods are used to determine the colour of the
   links.
  \item private static double maxcompare(double[][][] dist), which
  is used to draw the thickness of the links and
  returns $max_{k}[max_{i}(P(n \geq n_{i}|a_{k})-P(n \geq
  n_{k}|a_{o}))]$
\end{itemize}

\subsubsection{Dynamic explanation}

For the dynamic explanation, the main data structures are related
with the management of several evidence cases, which are:
\begin{itemize}
  \item The class \emph{Case}, whose main attributes are the probabilities of the
  nodes, the observed nodes, an attribute to know if it is going to be shown in the expanded nodes,
  its color and identifier and an attribute to know if the
  evidence of the case has been propagated.
  \item The class \emph{CasesList} represents a list of cases,
  defined mainly by a Vector of Cases. Moreover, it has
  some additional attributes to know which cases are the first and last to be
  displayed, the number of elements of the list and, what is very important, an attribute to know which is the current
  case, since the user can navigate through the list of cases.
  Each time some evidence is introduced, it is added to the current case, except when the current case is the prior case, in whose
  case, a new case is created and then it is the current case. The list is created as empty each time the user change from edit to
  inference mode. If the network has been compiled, it will also
  contain the \emph{prior case}. Then, in the class InferencePanel, the attribute \emph{private
CasesList cases} represents the list of cases created at each
moment.
  \item The class \emph{CaseEditor} defines a dialog which allows the user to edit and
  modify the evidence corresponding of a certain case and even to
  navigate through the list of cases. With such a purpose, it has
  methods for adding and removing findings to the evidence and to
  propagate it.
  \item The class \emph{CaseMonitor} also defines a dialog which
  provides the user the possibility of modify some properties of a
  certain Case, such as the colour, identifier, etc. It has
  methods also for adding a new case, for deleting some of them
  and even for explaining one selected case, as we will describe below.
\end{itemize}

The \textbf{explanation} of a Case of Evidence is managed through
the following classes:
\begin{itemize}
  \item \emph{ExplainCase}, it has specially implemented with the aim of
  providing some dynamic explanations for an specific evidence
  Case. It contains an inner class, ExplainingAction, which is in
  charged of generating the explanation by way of its methods \emph{whyButton\_actionPerformed
  (ActionEvent event)}, which classify the findings of the evidence case depending on the kind and amount of
  evidence they exert on a selected variable, and \emph{howButton\_actionPerformed (ActionEvent
  event)}, which shows the chains of reasoning from the evidence
  to a certain variable;
  \item \emph{EvidenceAnalysis} is a class for defining the dialog
  in which the classification of findings is displayed. For this
  purpose, it contains as attributes four vectors, one for each
  group of findings depending on the kind of influence (positive, negative, null, unknown) they
  exert. To know in which group each finding should be include,
  the method classifyFindings() calculate de cumulative
  distribution taking into account the whole evidence calling to
  the following methods of class \emph{macroExplanation}:
  \begin{itemize}
  \item public static double[] greaterdist(Case c, Node n), which
  returns the cumulative distribution of $n$ given the evidence of
  case $c$;
  \item public static int compare(double[] d, double[] t) returns 0,1 or 2 depending on whether $d$
  is less, greater or equal to $t$. If they can not be compared,
  it returns 3;
  \item public static double influences(double[] d, double[] t) returns the amount of difference among distributions $d$ and $t$.
  \end{itemize}
  Then, to know in which group a finding $f$ must be included,
  the method classifyFindings obtains in \emph{distnode} the cumulative distribution
  of the selected node given the whole evidence and then, in \emph{distfinding}, the cumulative
  distribution of the same node given the evidence except $f$. After comparing those distribution,
  if the result is 1 or 2, then the calling to influences gives
  the amount of influence.
  \item \emph{macroExplanation}, \emph{Node} and \emph{InferencePanel} contains some more
attributes and methods to display the chains of reasoning from the
evidence to a certain variable. The first one contains the method
public static NodeList pathExplanation(CasesList c, Bnet b,
Evidence e, Node h), which returns the nodes which appear in the
paths from the evidence of the current case of $c$ in the network
$b$ to node $h$, obtained by the method proposed by Suermondt
(1992). During the performing of that method, those nodes are
marked thanks to the attributes \emph{visited} and \emph{marked}
of class \emph{Node}. Moreover, the nodes in those paths are
coloured depending on the kind and amount of influence they
receive from the evidence. Then, in class \emph{InferencePanel} we
have added the attribute MACROEXPLANATION to know when a node must
be coloured.
\end{itemize}

Finally, the options for explanation can be set by the user thanks
to the dialog defined by the class \emph{OptionsExplanation}. The
main attribute of this class is a reference to the InferencePanel
in which the explanations are provided. This is because, as we
have already described, it has the information needed to generate
the explanations. In fact, its attribute \emph{public int
COMPARINGCASE} is defined to represent the case taken as reference
to colour the nodes. Also, \emph{private double Theta} defines the
threshold upon which the influence is relevant for the user and
\emph{public boolean AUTOEXPLANATION} is used to present graphical
explanations automatically.

\subsection{Other aspects of the Elvira GUI}
\subsubsection{Warnings and errors}
With the aim to help the user during the performance of Elvira, it
provides different warning and error messages depending on the
kind of situation. This is managed by the class
\emph{ShowMessages}, which includes several Strings representing
the identifiers of the different situations which can provoke an
execution error or generate a warning. Since all messages are
displayed in a dialog, all its methods are defined as static in
order to use the predefined class of Java \emph{JOptionPane},
which provides a method for showing an option dialog.

\subsubsection{Internationalization: Package
\emph{elvira.localize}}

Up to date, Elvira can be run in Spanish and (American)
English\footnote{The language selected by default is Spanish}.
Nevertheless, it is implemented to be easily extended to another
languages, thanks to the facilities for internationalization
provided by Java, what has been taken into account when designing
Elvira. The idea is not to include in the source code the original
text to be displayed in the GUI (what would imply to write it in
any possible language) but to define some ``text constants'' whose
translation in a specific language is stored in certain text
files, characterized by its extension: \emph{.properties}. As a
result, there would be one of these files for each different
language in which Elvira GUI is going to be provided. Thereby, the
source code only includes such ``text constants'', defined by the
programmer, and the name of the file where their corresponding
translations can be found, depending on the language selected by
the user. To easily know to which language corresponds each file,
we decided to add to their names the symbol \_ together with two
letters representing the language, except to the one for
(American) English. For example, \emph{Dialogs.properties}
contains the translations for every text displayed in the dialogs,
and \emph{Dialogs\_sp.properties}, the associated translations in
Spanish. Moreover, in order to avoid the generation of files too
large, we have classified them, depending on its purpose:
\emph{Dialogs*, Explanation*, Menus*} and \emph{Messages*}, all of
them included in the package \emph{elvira.localize}.
