\section{Introduction}
Expert systems based on graphical models are one of the most succesful and useful products 
emerged from Artificial Intelligence research. By using graphical models is possible to 
efficiently encode large joint probability distributions that would be unmanageable in other way. 
This is due to the fact that graphical models offer a language to describe a joint probability 
distribution in terms of local probabilistic dependences between events~\cite{jensen-book2001}. 
Moreover, if there exists enough conditional independences, the number of
required parameters is reduced to a reasonable size.\\

We deal with the task of representing compactly the belief obtained from the 
combination of the belief of a group of experts by means of a Bayesian network. A Bayesian network 
usually codifies one expert's subjective belief, otherwise, when building a greater model, we can 
consult with several experts, each one specialized in some knowledge subset of the whole domain. 
It would be desirable to combine all that knowledge provided under the form of Bayesian networks
into a single more general representation.\\

The fusion proces has two different phases. First, we combine independence graphs based on the 
union and intersection of independences of the initial models. Then, the combination of probability 
values is performend taking as basis the consensus graph. So, {\em qualitative aggregation} deals with 
the estimation of the consensus model's structure, whilst {\em quantitative aggregation} deals with 
the estimation of the consensus model's parameters. Thus, the order in which these types of 
aggregation are been performed, determines the main approaches that have been addopted to tackle 
with the Bayesian network agregation problem.



