%%%
% Procedimientos 'te�ricos' para el problema.
%
%    Algoritmo implementado en Elvira.
%
%    Psudoc�digo del algoritmo.
%
%    Referencias bibliogr�ficas.
%%%

\section{Qualitative aggregation}
There are several ways of combining directed independence graphs trying to
preserve their independences. The union of independence models can be achieve
under certain conditions by means of {\it graph intersection}. The intersection
of independence models can be obtained by means of {\it graph union}. And, finally,
we can combine the initial directed graphical models preserving a {\it maximal} 
number of independences. These qualitative aggregation methods are outlined in 
the next sections. An interested reader can find all the details in 

\subsection{Union of Independence Models}
We want to preserve the independence relationships of the original directed models. Let us 
assume that both models are defined over the same set of variables. If is possible to find an
ancestral ordering of the set of variables compatible with both models, then the {\it intersection
graph} is a minimal I-map of the independence model that is obtained by the union of the initial
models completed with the independence relationships given by the graphoid axioms. See \cite{CastilloGutHad97}
for details.

The sitiation changes if the initial models have only several variables in common. In this case, 
each initial model is modified (extended) to incorporate the proper variables of the other. Also, as in the
previous case, if is posible to find an ancestral ordering compatible with the variables of both models,
then the {\it intersection graph} is a minimal I-map of the independence model obtained by the 
union of the extended models completed with the independence relationships given by the
graphoid axioms. Note that the resulting directed acyclic graph will probably have several 
unconnected variables. This problem can partially be avoided if we extend the intersection graph
adding all the arcs that come from a proper variable to another proper variable or to a common one.
This is what we call {\it extended intersection}.  

The main drawback of union of independence models rest in the great number of dependences (arcs)
eliminated, yielding an structure strongly unconnected.

\subsection{Intersection of Independence Models}
Now, we want to preserve the independence relationships shared by the original directed models. 
If both models are defined over the same set of variables and exists an
ancestral ordering of the variables compatible with both models, then the {\it union
graph} is a minimal I-map of the independence model defined by the intersection of the initial
models.

In the case that the initial models have only several variables in common, we will have to 
extend each one of the initial models adding the proper variables of the other. Also, as in the
previous case, if there exists a compatible ancestral ordering with resptect to both models,
then the {\it union graph} is a minimal I-map of the independence model obtained by the 
intersection of the extended models.

The main drawback of intersection of independence models rest in the great number of 
dependences (arcs) preserved, yielding an structure strongly connected.

\subsection{Maximal Independence Models}
The idea behind a {\it maximal independence model} is combine models sharing some variables
but preserving as many original independencies as possible whilst the interactions produced
between the models are mediated by the common variables. See \cite{SagradoMoral2001} for details.

Suppose that the initial models are two directed acyclic graphical model $G_1 = (V_1, A_1)$ and
$G_2 = (V_2, A_2)$. Now, from these models we want to build a directed maximal graphical model $G$.
But, before going into a detailled description of this process we need to define the concept of 
{\it marginalization of a directed graphical model}.
Let $G = (V,A)$ be a directed graphical model and $\sigma$ an ancestral ordering of its nodes. The 
result of marginalizing $G$ to a subset $S \subseteq V$ is a new directed graph $G^{\downarrow S} = (S, A_S)$
such that for all $v \notin S$ we will add two sets of arcs:
\begin{itemize}
\item[i)] $\{(x,y) \vert \forall x,y \in S, x \in Pa(v), y \in Ch(v) \}$
\item[ii)] $\{(x,y) \vert \forall x,y \in S, x,y \in Ch(v), x \neq y, \sigma(x) < \sigma(y) \} $
\end{itemize}
where $Pa(v)$ and $Ch(v)$ represents the sets of direct predecessor and direct successors of $v$,
respectively.
It is possible to find a directed maximal acyclic graph combining $G_1$ and $G_2$, if there
exists an ancestral ordering $\sigma$ common to both graphs, following the next process:
\begin{enumerate}
\item Let $V = V_1 \cap V_2$ be the set of shared variables between $G_1$ and $G_2$.
\item Compute $G_2^{\downarrow V}$. Let $G_1^T$ be the graph obtained deleting from $G_1$ all arcs
contained in $G_2^{\downarrow V}$.
\item $G' = G_1^T \cup G_2$.
\item Compute $G_1^{\downarrow V}$. Let $G_2^T$ be the graph obtained deleting from $G_2$ all arcs
contained in $G_1^{\downarrow V}$.
\item $G'' = G_1 \cup G_2^T$.
\item Return the graph $G = G' \cap G''$ as the graphical maximal model.
\end{enumerate}

\section{Quantitative aggregation}
In many situations, a designer of a probabilistic system has to consult
more than one expert. If each one of the $n$ consulted experts holds a subjective belief 
expressed under the form of a joint probability distribution $P_i$, then a consensus joint
probability distribution $P$ is any function $f$ of the $P_i$,
$$
P \equiv f(P_1, \cdots, P_n),
$$
where $P$ is itself a legal joint probability distribution and $f$ is the {\it aggregation} or
{\it combination function}. 

Aggregation functions also deserve a more detailed attention, because they define the aggregation 
scheme used to combine experts' opinions. The combination functions implemented in Elvira are:
\begin{itemize}
\item{Linear opinion pool} The consensus probability is a weighted sum of all the probabilities
that various experts have assigned to it
$$
P(x_j) = k * \sum_{i=1}^n {w_{ij}} \cdot (P_i(x_j)).
$$
where $k$ is a normalization constant. According to this scheme, weights assigned to each 
hypothesis by experts must be normalized and must add up to one ($\forall j, \sum_{i=1}^m w_{ij}=1$),
and may not be negative ($\forall i,j, w_{ij} \ge 0 $).

\item{Logarithmic opinion pool} Here, the consensus probability assigned to $x_j$ is a 
weighted product of all the probabilities that the different experts have assigned to it
$$
P(x_j) = k * \prod_{i=1}^n (P_i(x_j))^{w_{ij}}.
$$
where $k$ is a normalization factor and the {\it experts' weights}, $w_{ij}$, as in the 
previous aggregation function are nonnegative numbers that must be normalized and must
add up to one. 
\end{itemize}

Observe the influence that the different scales each subject uses to express his/her personal 
degrees of belief have in the consensus. Whereas linear opinion pool depends on the assumption
that all the opinios are expressed using the same probability scale, logarithmic opinion pool
is invariant under rescaling of individual degrees of belief.

Another problem to asses the numerical parameters of a Bayesian network arises when we have to
build a conditional probability distribution for a node having more than two or three parents.
Elvira permits the use of the Noisy Or model to represent a probabilistic relationship among a
finite number of variables simplifying the estimation of probability values. The causal interpretation
of this model is that the effect $E$ is the result of a combination of intermediate events $Z_i$
which are caused by $C_i$ the parents (causes) of $E$. In this case, we have a contitional 
distribution $P(z_i \vert c_i)$ representing the relationship between $C_i$ and $Z_i$, whilst
the interaction between causes $C_i$ and efect $E$ is given by the logical OR. Thus, the conditional
probability $P(e \vert {\bf c})$ is obtained as
$$
P(e \vert {\bf c}) = \sum_{{\bf z} \vert OR({\bf z})=e} \prod_i P(z_i \vert c_i).
$$
To cover situations in which is unfeasible to identify all the variables $C_i$ that have a direct
influence over $E$, usually is included a "leak" probability term $P(e_L) = P(e \vert \neg c_1, \cdots, \neg c_n)$.
This term represents the situation in which the effect $E$ may occur even when all of its causes
are inactive and it is incorpored in the noisy Or model producing the leaky OR model that is defined
as
$$
P(e \vert {\bf c}) = \sum_{{\bf z} \vert OR({\bf z})=e} \prod_i P(z_i \vert c_i) \cdot \sum_{e_L \vert OR({\bf z}, e_L)}P(e_L).
$$
The main advantage of the Or models is the reduction, from an exponential
to a linear order of magnitude, of the number of parameters necessary to
specify the conditional probability distribution $P(e \vert {\bf c})$. Also,
the parameters  lend themselves to more intuitive interpretations, simplifying the 
construction of a probabilistic network model.

{\it Elvira} includes several facilities for performing the previously described aggragation
processes. These facilities can easily been accesed either by the graphical user interface or
by on-line command instantiation of the specific {\it Elvira} classes. The next sections explains
how to use the fusion tools inside {\it Elvira} software. 

%%%
% Estructura de clases Principales para resolver el problema. -
%
% Uso del algoritmo.
%
%    Uso en l�nea de comandos como API
%
%    Uso desde el interface gr�fico
%%%

%\section{Algorithms for Fusion}

\section{Fusion from the Elvira GUI}
This section is devoted to explain the capabilities related with the combination of Bayesian
networks available from the {\it Elvira Graphical User Interface} (Elvira GUI). First of all,
the user have to be using at least two Bayesian networks. This could be done editing two
completely new networks or opening {\tt elv} files containing their definition. In that moment, 
will be activated the option {\it Task} in the menu bar. Select {\it Task -- Fusion} to 
display the {\it Fusion options} pop-up wimdow. Here, the user can select from the
{\it list of active networks} those to be use in the fusion process. Also, he/she has to select,
respectively, the quantitative and qualitative types to apply from the corresponding fold-down 
field. Finally, click on the button {\it Fusion} in order to {\it Elvira} computes the 
combination network. The result is displayed in a new window. The user can quit fusion
process at any time by clicking on the button {\it Cancel}, but she/he will loose all the
given options.

\section{Fusion from the Command Line}
The {\tt Fusion} class of {\it Elvira} encloses the principal methods to perform the aggregation 
methods described. These are
\begin{itemize}
\item {\tt public void linearPool (int qualitativeType, Bnet Bn1, Bnet Bn2)} Computes the
linear pool of two Bayesian networks. The value assigned to {\tt qualitativeType} indicates 
the type of qualitative fusion that is going to be applied in order to obtain the structure of 
the fusion network. Thus, a value of $0$ is used for graph union, $1$ for graph intersection,
$2$ for extended graph extended and $3$ for the maximal graph. {\tt Bn1} and {\tt Bn2}
specify the Bayesian networks objects that are going to be combined.
\item {\tt public void logarithmicPool (int qualitativeType, Bnet Bn1, Bnet Bn2)} Computes the
logarithmic pool of two Bayesian networks objects {\tt Bn1} and {\tt Bn2}. As in the previous 
method {\tt qualitativeType} indicates the type of qualitative fusion can take the same mentioned 
values.
\item {\tt public void noisyORPool (int qualitativeType, Bnet Bn1, Bnet Bn2, boolean loose)}
This method must be used if the parameters of all the variables in the combined network are going to 
be estimated following a noisy Or model. The parameter {\tt loose} indicates the use of a 
leaky probability term. Also, as in the previous models the combined structure is computed
applying the corresponding {\tt qualitativeType}. 
\item {\tt public void noisyORPool (int qualitativeType, Bnet Bn1, Bnet Bn2, NodeList ORnodes, boolean loose)}
In the general case, once the qualitative combination determines the common structure, we have only 
to estimate the conditional probabilities associated to a few variables, pointed out as {\tt ORnodes}, 
by means of a noisy Or model. The rest of the parameters are computed applying a linear pool.
\item {\tt public Fusion(int qualitativeType, int quantitativeType, Bnet bN1, Bnet bN2)}
This constructor creates a new {\tt Fusion} object representing the Bayesian network 
computed from the fusion of another two networks. The user has to indicate the 
{\tt qualitativeType} used to build the common structure and the {\tt quantitativeType} of
aggregation used to combine the parameters. The values allowed for the {\tt quantitativeType}
are $0$ for linear pool, $1$ for logarithmic pool, $2$ noisy Or and $3$ noisy Or with loose
probability.
\end{itemize}

The {\tt Fusion} class also includes some methods that can be classified as utilities, such as
\begin{itemize}
\item {\tt public Vector compare (Bnet Bn)} Compares the actual network with another Bayesian 
network. The comparison is done counting coincident links, inverted links, added links and 
computing the Kullback-Leibler divergence between the joint distributions. The unique requisite 
between the networks being compared is that they have the same set of variables.
\item {\tt public Vector split (int pcNodes, int pcShared)} Splits a Bayesian network. 
Creates a {\tt Vector} object containing two new Bayesian networks each one with a given
percentage ({\tt pcNodes}) of the original nodes and sharing a percentage ({\tt pcShared}) of 
common nodes. The sum of the percentage of nodes and half of the percentage of shared nodes 
can not be greater than $100$.
\item {\tt public Vector randomSplit (int pcNodes, int pcShared)} Splits randomly a Bayesian network. 
Creates a vector containing two new marginalized Bayesian networks with a percentage ({\tt pcNodes})
of the original nodes and sharing a percentage ({\tt pcShared}) of nodes.
\end{itemize}

{\it Elvira} also includes the class {\tt Fuse} in order to perform easily fusion experiments from the 
command line. The syntax to invoke this class is
\begin{quote}
{\tt java Fuse qualitative\_type quantitative\_type input\_file\_1 input\_file\_2 [output\_file]}
\end{quote}
where {\tt qualitative\_type} and {\tt quantitative\_type} have to be replaced with a number in the 
range between $0$ and $3$ indicating, respectively, the qualitative and quantitative aggregation 
methods that are being used. The values allowed are $0$ for graph union and linear pool, $1$ for
graph intersection and logarithmic pool, $2$ for extended graph intersection and noisy Or estimation, 
and $3$ for the maximal graph and noisy Or with loose probability estimation. It is mandatory
to indicate the names of the two input files containing the networks that are going to be combined,
by the way these are {\tt elv} files. Optionally, the user can type the name where the resulting
Bayesian network has to be stored. By default, the program stores the combination result in a 
file called {\tt fuse.elv}.