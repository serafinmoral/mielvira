\typeout{>>>>>>>>>>>>>>>>>>>>>>>>>>>>>>>>>>>>>>>>>>>>>>>>>>>>>>>>>>>>>>>>>>>>>>>>>>>>}
\typeout{>>>>>>>>>>>>>>>>>>>>>>>>>>>>>>>>>>>>>>>>>>>>>>>>>>>>>>>>>>>>>>>>>>>>>>>>>>>>}
\typeout{>>>> Comandos.tex de Elvira-Book contiene comandos que pueden ayudar a}
\typeout{>>>> simplificar la escritura. }
\typeout{>>>> Incluye los que consideres necesarios. No olvides comentarlos}
\typeout{>>>> CUIDADO: No uses nunca "renewcommand"}
\typeout{>>>>}


%%% Se muestran algunos ejemplos que pueden ser �tiles.
%%% Se han dividido por 'secciones' y cada uno de ellos tiene un comentario asociado

%%%%%%%%%%%%%%%%%%%%%%%%%%%%%%%%%%%%%%%%%%%%%%%%
%%%%%%%%%%% NO TOCAR ESTOS COMANDOS.
\newcommand{\elvira}[1]{\textsf{#1}} % \textsf para las palabras claves de Elvira.
\newcommand{\autores}[1]{\addtocontents{toc}{\protect\textsl{#1}}} % Autores en el �ndice
\newcommand{\autor}[3]{\centerline{\LARGE #1} \centerline{#2} \centerline{#3} ~\\ } % Autores en Chap

%%%%%%%%%%%%%%%%%%%%%%%%%%%%%%%%%%%%%%%%%%%%%%%%

%%%%%%%%%%%%%%%%%%%%%%%%%%%%%%%%%%%%%%%%%%%%%%%
%%%%%%%% Comandos para hacer Cajas %%%%%%%%%%%%
%%%%%%%%%%%%%%%%%%%%%%%%%%%%%%%%%%%%%%%%%%%%%%%

% Para enmarcar un tabular
\newcommand{\ffbox}[1]{\fbox{\begin{tabular}{c} #1 \end{tabular}}}



%%%%%%%%%%%%%%%%%%%%%%%%%%%%%%%%%%%%%%
%%%%%%%% Graficos-Flechas %%%%%%%%%%%%
%%%%%%%%%%%%%%%%%%%%%%%%%%%%%%%%%%%%%%
\newcommand{\da}{\downarrow}    % Flecha abajo
\newcommand{\ua}{\uparrow}      % Flecha arriba
\newcommand{\la}{\leftarrow}    % Flecha izda.
\newcommand{\ra}{\uparrow}      % Flecha dcha

\newcommand{\lineah}{\relbar\joinrel} % Realiza una linea en horizontal. Auxiliar para los siguientes.
\newcommand{\mapto}{\mapstochar\lineah\joinrel\lineah\joinrel\rightarrow} % Para funciones
\newcommand{\Mapto}{\mapstochar\lineah\joinrel%
       \lineah\joinrel\lineah\joinrel\rightarrow}               % Igual que mapto pero m�s largo.
\newcommand{\longmapto}{\mapstochar\lineah\joinrel\lineah%
           \joinrel\lineah\joinrel\lineah\joinrel\rightarrow}   % Igual que Mapto pero m�s largo.
\newcommand{\LongMapto}{\mapstochar\lineah\joinrel\lineah\joinrel\lineah%
           \joinrel\lineah\joinrel\lineah\joinrel\rightarrow}   % Igual que longmapto pero m�s largo.




%%%%%%%%%%%%%%%%%%%%%%%%%%%%%%%%%%
%%%%%%%% Caligr�ficos %%%%%%%%%%%%
%%%%%%%%%%%%%%%%%%%%%%%%%%%%%%%%%%

\newcommand{\cA}{{\cal A}}  \newcommand{\cB}{{\cal B}}  % Paa no tener que escribir siempre \cal
\newcommand{\cC}{{\cal C}}  \newcommand{\cD}{{\cal D}}
\newcommand{\cE}{{\cal E}}  \newcommand{\cF}{{\cal F}}
\newcommand{\cG}{{\cal G}}  \newcommand{\cH}{{\cal H}}
\newcommand{\cI}{{\cal I}}  \newcommand{\cJ}{{\cal J}}
\newcommand{\cK}{{\cal K}}  \newcommand{\cL}{{\cal L}}
\newcommand{\cM}{{\cal M}}  \newcommand{\cN}{{\cal N}}
\newcommand{\cO}{{\cal O}}  \newcommand{\cP}{{\cal P}}
\newcommand{\cQ}{{\cal Q}}  \newcommand{\cR}{{\cal R}}
\newcommand{\cS}{{\cal S}}  \newcommand{\cT}{{\cal T}}
\newcommand{\cU}{{\cal U}}  \newcommand{\cV}{{\cal V}}
\newcommand{\cX}{{\cal X}}  \newcommand{\cY}{{\cal Y}}
\newcommand{\cZ}{{\cal Z}}

\newcommand{\N}{\mathbb{N}} % Conjunto naturales.
\newcommand{\Z}{\mathbb{Z}} % Conjunto enteros.
\newcommand{\Q}{\mathbb{Q}} % Conjunto racionales.
\newcommand{\R}{\mathbb{R}} % Conjunto reales.



%%%%%%%%%%%%%%%%%%%%%%%%%%%%%%%%%%%%%%%%%%%%%%%%%%%%%%%%%%%%%%%%%%%%%%%%%
%%%%%%%% Simplifica la escritura y hace m�s legible el texto %%%%%%%%%%%%
%%%%%%%%%%%%%%%%%%%%%%%%%%%%%%%%%%%%%%%%%%%%%%%%%%%%%%%%%%%%%%%%%%%%%%%%%
\newcommand{\mc}{\multicolumn}
\newcommand{\ds}{\displaystyle}
\newcommand{\bs}{\boldsymbol}
\newcommand{\fdNormal}[3]{% P.e. $\fdNormal(x,\mu,\xi)$ escribe una N(\mu,\xi) en funci�n de x
    f(#1)={1\over \sqrt{2\pi} #3}\exp\left\{{-{1\over 2}{(#1-#2)^2 \over #3^2}}\right\}
}

%%%% A�ade los que necesitas

%%%% A�adidos por Antonio
\newcommand{\bx}{\mathbf{x}}
\newcommand{\by}{\mathbf{y}}
\newcommand{\bz}{\mathbf{z}}
\newcommand{\be}{\mathbf{e}}
\newcommand{\bX}{\mathbf{X}}
\newcommand{\bY}{\mathbf{Y}}
\newcommand{\bZ}{\mathbf{Z}}
\newcommand{\bE}{\mathbf{E}}
\def\set#1{{\{#1\}}}
\newcommand{\flechab}[1]{^{\downarrow #1}}
\newcommand{\flechaa}[1]{^{\uparrow #1}}

%%%%%%%%%%%%%%%%%%%%%%%%%%%%%%%%%%%%%%%%%%%%%%%%%%
\typeout{<<<<}
\typeout{Cargados todos los nuevos Comandos de Elvira-Book}
\typeout{<<<<<<<<<<<<<<<<<<<<<<<<<<<<<<<<<<<<<<<<<<<<<<<<<<<<<<<<<<<<<<<<<<<<<<<<<<<<}
\typeout{<<<<<<<<<<<<<<<<<<<<<<<<<<<<<<<<<<<<<<<<<<<<<<<<<<<<<<<<<<<<<<<<<<<<<<<<<<<<}
