\typeout{>>>>>>>>>>>>>>>>>>>>>>>>>>>>>>>>>>>>>>>>>>>>>>>>>>>>>>>>>>>>>>>>>>>>>>>>>>>>}
\typeout{>>>>>>>>>>>>>>>>>>>>>>>>>>>>>>>>>>>>>>>>>>>>>>>>>>>>>>>>>>>>>>>>>>>>>>>>>>>>}
\typeout{>>>> Paquetes.tex de Elvira-Book contiene paquetes referentes a}
\typeout{>>>> Lenguaje, Fuentes, Tipograf�a, Graficos, Herramientas, Hipertexto, ...}
\typeout{>>>>}

%%%%%%%%%%%%%%%%%%%%%%%%%
%%%%% Paquetes a Utilizar en el documento
%%%%%%%%%%%%%%%%%%%%%%%%%
% Permite poner bibliograf�as en cap�tulos a partir de una  �nica .bib
\usepackage[sectionbib]{chapterbib} % Debe ir antes de babel
% Lenguaje
\usepackage[english]{babel}

% Fuentes y Tipograf�a
\usepackage[T1]{fontenc}
%\usepackage[latin1]{inputenc}%Espa�ol
%\usepackage{cmbright}%\usepackage{avantgar}
%\usepackage{palatino}
%\usepackage{times}
%\usepackage{helvet}
\usepackage{amsmath,amssymb,amsfonts}


%Gr�ficos
\usepackage{graphicx,epic,eepic} % Para incluir gr�ficos de Xfig
\usepackage{pst-all,pstricks}    % Para poder dar color al texto.
\usepackage{subfigure}

%Herramientas
\usepackage{listings} %Para hacer listados de Java
\usepackage{multirow} %Varias filas en una tabla
\usepackage{tabularx} % Varias lineas en una celda
\usepackage{rotating} % Para Rotar el texto
\usepackage{ecltree} % Para hacer �rboles
\usepackage{paralist} % Hacer listas de una forma m�s flexible




%Presentaci�n del documento
\usepackage[pagestyles]{titlesec}   % Permite definir formatos de cap�tulos, secciones, etc
                                    % Tambi�n para cabeceras y pies de p�gina. Es m�s flexible que el fancyhdr.
                                    % Las definiciones del paquete Presentacion.tex

\typeout{<<<<}
\typeout{Cargados todos los Paquetes de Elvira-Book}
\typeout{<<<<<<<<<<<<<<<<<<<<<<<<<<<<<<<<<<<<<<<<<<<<<<<<<<<<<<<<<<<<<<<<<<<<<<<<<<<<}
\typeout{<<<<<<<<<<<<<<<<<<<<<<<<<<<<<<<<<<<<<<<<<<<<<<<<<<<<<<<<<<<<<<<<<<<<<<<<<<<<}
