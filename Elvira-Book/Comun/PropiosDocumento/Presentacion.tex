\typeout{>>>>>>>>>>>>>>>>>>>>>>>>>>>>>>>>>>>>>>>>>>>>>>>>>>>>>>>>>>>>>>>>>>>>>>>>>>>>}
\typeout{>>>>>>>>>>>>>>>>>>>>>>>>>>>>>>>>>>>>>>>>>>>>>>>>>>>>>>>>>>>>>>>>>>>>>>>>>>>>}
\typeout{>>>> Presentacion.tex de Elvira-Book define la apariencia del documento}
\typeout{>>>> simplificar la escritura.
CUIDADO: No uses nunca "renewtheorem"} \typeout{>>>>}

%Fuentes -> Introducelas al comienzo del fichero Paquetes.tex

%Profundidad de la tabla de contenidos
    \setcounter{secnumdepth}{4}
    \setcounter{tocdepth}{4}


%Tama�o Papel
        %%%%% Descomentar la siguiente l�nea para la versi�n final
    %\usepackage[a4paper]{geometry}
        %%%%% Definimos el maximo tama�o posible mientras se hagan versiones.
        %%%%% Comentar las siguientes l�neas  para la versi�n final.
        \marginparwidth 0pt     \marginparsep 0pt
        \topmargin   0pt        \textwidth   6.5in
        \textheight 23cm
        % Margen izq del txt en impares.
        \setlength{\oddsidemargin}{.0001\textwidth}
        % Margen izq del txt en pares.
        \setlength{\evensidemargin}{-.04\textwidth}
        % Anchura del texto
        \setlength{\textwidth}{.99\textwidth}



%Para los T�tulos e �ndices iniciales
    %OJO!! Introducir \usepackage{titlesec} en Paquetes.tex

\titleformat{\chapter}[display]
    {\normalfont\Large\filcenter\sffamily}
    {\titlerule[1pt]%
     \vspace{2pt}%
       \titlerule\LARGE\MakeUppercase{\chaptername} \thechapter}
    {1pc}
    {%\titlerule
     \vspace{1pc}
     \Huge}
     [\vspace{1pc}%
     \titlerule%
     \vspace{1pt}%
     \titlerule%
     \color{black}]



\newpagestyle{main}[\small\sffamily]{
\headrule
\footrule
\sethead [\textbf{\thepage}]
[\textsl{\chaptertitle}]
[Chapter \toptitlemarks\thechapter]
{Section \toptitlemarks\thesection}
{\textsl{\sectiontitle}}
{\textbf{\thepage}}
\setfoot[]%
        [\small \textcolor{red}{\textbf{Elvira Consortium - DRAFT}}]%
        []
        {}%
        {\small \textcolor{red}{\textbf{DRAFT}}}%
        {}
}
\pagestyle{main}

\typeout{<<<<}
\typeout{Definida la Presentaci�n de Elvira-Book}
\typeout{<<<<<<<<<<<<<<<<<<<<<<<<<<<<<<<<<<<<<<<<<<<<<<<<<<<<<<<<<<<<<<<<<<<<<<<<<<<<}
\typeout{<<<<<<<<<<<<<<<<<<<<<<<<<<<<<<<<<<<<<<<<<<<<<<<<<<<<<<<<<<<<<<<<<<<<<<<<<<<<}
