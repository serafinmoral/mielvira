\section{Introduction}
Data quality is a key issue in Data Mining; if data quality is not good
enough algorithms can not reveal all the information contained in the data
and revealed information can not be as precised as expected. In the real world,
databases lack off some non-desirable characteristics such as:

\begin{itemize}
    \item   Incomplete: Missing Values.
    \item   Noisy: Errors or Outliers.
    \item   Inconsistent: Discrepancies or Inconsistent Names.
\end{itemize}

On the other hand there are other characteristics that although they are not been
as negative as the previous one, can lead to an efficiency reduction or 
make analysis difficult e.g. a database with thousands of variables
(very common in bioinformatics).

Data nature can also be a key issue because data can imply the use of 
some specific kind of algorithms to perform specific operations on data and
not any other. This problem is addressed when algorithms are not as efficient as
expected or these can requiere too much CPU. This situation leads to the fact some 
transformations should be made to data with an information loss.

But there exist other kind of problems related to the relation between 
different databases. This fact can generate some problematic situations:
\begin{itemize}
    \item   The same concept expressed differently in each database.
    \item   The same attribute can contain values expressed in different ways.
    \item   The same attribute name means different things.
    \item   \ldots
\end{itemize}

So in these conditions a preprocessing task must be performed in order to increase
data mining technique efficiency. It is very typical that preprocessing steps can
require much more effort than our initial target.To solve, this each problem has 
been identified, studied and analyzed leading to the birth of some typical tasks.

Preprocessing tasks can be grouped in:
\begin{itemize}
    \item   \textbf{Data Cleaning}: missing values are handled (e.g. using most probable
    		value based on existing data), outliers are removed (by means of histograms,
    		cluster analysis, or regression) and incosistents are solved.
    \item   \textbf{Data Integration}: integration and consolidation of different sources into
    		only one repository.
    \item   \textbf{Data Normalization}: scaling data values in a range such as $[0..1]$ or 
    		$[-1..+1]$ can prevent outweighting features.
    		Typical methods are:
    		\begin{itemize}
    			\item	Min-Max: 
				\begin{eqnarray}
					y'=\frac{(y-max)}{max-min}(max'-min')+min' 
				\end{eqnarray}
				where $max'=1$ and $min'=-1$ or $min'=0$.
    					
    			\item	Z-Score:
				\begin{equation}
					y'=\frac{y-\mu}{\sigma}
				\end{equation}
    			\item	Decimal Scaling: 
				\begin{eqnarray}
					y'=\frac{y}{10^{n}} \text{ n is the number of digits.}
				\end{eqnarray}
    		\end{itemize}
		in any case $y$ is the original data and $y'$ is the scaled data.
    \item   \textbf{Data Reduction}: increasing the efficiency reducing the huge data set to 
    		a smaller representative.
    		Typical methods are:
    		\begin{itemize}
    			\item	Discretization.
    			\item	Filtering Measures.
    			\item	Data Aggregation.
    			\item	Data Compression.
    			\item	Dimension/attribute reduction.
    		\end{itemize}
\end{itemize}

Elvira is able to perform the most typical tasks in preprocessing:
\begin{itemize}
    \item   Discretization
    \item   Filtering Measures.
    \item   Imputation
    \item   Aggregation
\end{itemize}
