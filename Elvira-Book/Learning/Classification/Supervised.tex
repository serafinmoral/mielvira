\section{Supervised Classification}

{\em Elvira} encloses five classic and well-known supervised
classification algorithms which can be acceded by the graphical
interface and the on-line command tool. The algorithms are the
naive Bayes \cite{cestnik90}, the selective naive Bayes
\cite{langley94}, tree-augmented naive Bayes (TAN)
\cite{friedman97}, the semi naive Bayes algorithm \cite{pazzani95}
and the {\em k}-dependence Bayesian classifier (KDB)
\cite{sahami96}. Coupled with the induction processes of these
classifiers, {\em Elvira} encloses several facilities which are
essential in a complete supervised classification study such as:
\begin{itemize}
\item estimation of the predictive accuracy percentage of the
classifier; \item display of the confusion matrix derived from the
previous estimation process; \item after the classifier is
learned, categorization of a datafile of unlabelled set of cases
with the class value predicted by the model for each case; \item
after the classifier is learned, given an unseen dataset of
labeled test cases, check the (in)equality of the given label and
the class value predicted by the model.
\end{itemize}

Two possibilities exist in the graphical interface of {\em Elvira}
to easiliy access these facilities: \begin{enumerate} \item The
third button of the icons-row of the graphical interface, that is,
the {\em Options for the database cases file} button, allows the
usage of these facilities. \item The facilities are also
accessible by clicking in {\em File} $\rightarrow$ {\em Open cases
file}. \end{enumerate}

These two possibilities open an user-friendly window which shows
the {\em machine learning} and {\em post learning} options to
handle the desired data and the induced models.

{\em Elvira} has a set of several Java classes which implement
these classifiers and facilities.

The {\em ClassifierValidator} and {\em  ConfusionMatrix} classes
deal with the accuracy estimation process and the calculation of
the associated confusion matrix, respectively. Both classes are
located in the {\em elvira.learning.classification} package.

By means of the {\em ClassifierValidator} class, {\em Elvira}
allows the hold-out, {\em k}-fold cross-validation and
leave-one-out validation schemes. The next lines are used to
describe the principal methods associated to the {\em
ClassifierValidator} class:

\begin{itemize}
\item public ClassifierValidator(Classifier classifier, Vector
dbcs, int classvar, int method); This class builder fixes the
classifier to be validated, a vector of {\em DataBaseCases}-type
objects, the position of the class label in the order of the
variables in the dataset, and the specific validation method to be
used. When a hold-out validation scheme is used, the {\em dbcs}
parameter hosts two {\em databasecases} objects, which represent
the training and testing subsets of cases, respectively; when {\em
k}-fold cross-validation or leave-one-out validation procedures
are used, the {\em dbcs} parameter represents a single set of
cases. \item public void setSeed(long seed); In order to deal with
the needed validation procedure, this method sets a new seed to
perform the needed random instance partitions in the dataset.
\item public Vector trainAndTest(); This method implements a
hold-out validation procedure, returning in the {\em Vector}
object the confusion matrices in the training and test subsets.
\item public ConfusionMatrix kFoldSumCrossValidation (int k); This
method implements a {\em k}-fold cross-validation procedure,
returning the confusion matrix associated to the validation
process. The returned confusion matrix hosts the sum of {\em k}
confusion matrices generated in the {\em k} partitions of the {\em
k}-fold cross-validation process. \item public Vector
kFoldCrossValidation\_Vector (int k); This method returns in a
{\em Vector} object the {\em k} confusion matrices generated in
the {\em k} partitions of the {\em k}-fold cross-validation
process. \item public ConfusionMatrix leaveOneOutSum(); This
method implements a leave-one-out procedure, returning the
confusion matrix associated to the validation process, which hosts
the sum of the confusion matrices generated for each instance of
the leave-one-out accuracy estimation process. \item public Vector
leaveOneOut\_Vector(); This method returns in a {\em Vector}
object the confusion matrices generated for each instance of the
leave-one-out accuracy estimation process. \item public double
error (Classifier classifier, DataBaseCases dbc, int classnumber);
This method computes the error rate for a classifier in a given
{\em DataBaseCases} object to be categorized. The {\em
classnumber} object indicates the position of the class label in
the order of the variables in the dataset. \item public
ConfusionMatrix confusionMatrix (Classifier classifier,
DataBaseCases dbc, int classnumber); Using the same arguments as
the previous method, this {\em ConfusionMatrix} method returns the
confusion matrix associated to the exposed categorization process
over a {\em DataBaseCases} object.
\end{itemize}

By means of the {\em ConfusionMatrix} class, {\em Elvira} allows
the display of the confusion matrix which is the consequence of
the used accuracy estimation scheme. The next lines are used to
describe the principal methods associated to the {\em
ConfusionMatrix} class:

\begin{itemize}
\item public ConfusionMatrix(FiniteStates classVariable); This
class builder has as parameter the class label itself. \item
public ConfusionMatrix(int classNumber); This class builder has as
parameter the number of different values of the class label. \item
public double getError(); This method prints in the command line
the error rate stored in the confusion matrix; \item public double
getStandardDeviation(); This method prints the standard deviation
stored in the confusion matrix; \item public void print(); By
means of this method, the confusion matrix is printed in the
command line.
\end{itemize}

The {\em Naive\_Bayes, SelectiveNaiveBayes,
WrapperSelectiveNaiveBayes, TAN, CMutInfTAN, SemiNaiveBayes,
WrapperSemiNaiveBayes, KDB} and {\em CMutInfKDB} classes deal with
the induction processes of the different classification models
from a dataset of labeled cases. Four pairs of these classes
collaborate in the construction and use of the mentioned
classification models. These classes are located in the {\em
elvira.learning.classification.supervised.discrete} package. In
order to induce and manage these supervised classifiers, and
located in the same package as previous classes, the {\em
DiscreteClassifier} abstract class hosts the general
characteristics and methods associated to a common supervised
classification process. Thus, the exposed classes associated to
the induction of different supervised classifiers, are children of
this abstract class. The rest of this section is used to present
the principal methods associated to these classes, associated with
the exposed classifier structures.

{\em DiscreteClassifier} is an abstract class, parent of the rest
classes which finally implement each of the supervised discrete
classifiers. (Note that these classes not allow the use of
continuous predictive variables.) This class hosts the facilities
and procedures which are general to any supervised classification
task (several of them were described in the previous paragraphs).
The following lines are used to describe the principal methods of
this general-purpose supervised-classification class:

\begin{itemize}
\item public DiscreteClassifier(); It is an empty builder for the
class. \item public DiscreteClassifier(DataBaseCases data, boolean
lap); This class builder fixes the dataset of cases to be used and
the possibility to use the Laplace correction in the learning of
the probabilities. The builder also checks whether the variables
of the dataset have discrete values. \item public void
setClassifier(Bnet model); This method is used to assign the
Bnet-type object as a classification model. \item public Bnet
getClassifier(); This method returns a classification model as a
Bnet-type object. \item public abstract void structuralLearning();
This abstract method, which is implemented in the subclasses,
learns the graphical structure of a model. However, it does not
learn the associated (conditional) probability distribution
tables. \item public void parametricLearning(); This method is
implemented in the subclasses of DiscreteClassifier class. It
fills, based on the dataset given to the class builder, the
(conditional) probability distribution tables associated to the
graphical structure learned by the previous structuralLearning()
method. The (conditional) distribution tables appear in the BNet
object. \item public void train(); In order to learn a classifier,
this method calls, in the given order, both structuralLearning()
and parametricLearning() methods. \item public void
learn(DataBaseCases data, int classIndex); The result of this
method is the same as first using the non-empty class builder,
followed by the train() method. \item public int assignClass
(double[] caseTest); Given the array object caseTest (which
represents an unlabeled instance), this methods returns, according
to a classifier previously learned by the train() method, the most
likely class label. \item public int assignClass(Configuration
conf); This method extracts an array of double-type numbers from
the given Configuration-type object, and it calls the previous
\\assignClass(double[] caseTest) method. \item public double
test(DataBaseCases test); Once a classifier has been learned, this
method tests its accuracy over a labeled set of examples given in
the DataBaseCases-type object. A confusion matrix is also
returned. \item public void categorize(String inputFile, String
outputFile); Once a classifier has been learned, this method can
be used to categorize (predict the most likely label value) a
datafile of cases given in the {\em inputFile} object. The
predicted class value is written as the last variable of an
example, that is, at the end of each line-sample. The categorized
file is returned in the {\em outputFile} object. \item public
ConfusionMatrix getConfusionMatrix(); This method, by means of the
{\em test(data)} method, returns a confusion matrix. \item
abstract Vector classify(Configuration instance, int classnumber);
It returns an ordered vector which collects, for a given {\em
instance}, the probability belonging to each class of the problem.
Note that it is assumed that the first class of the problem is
labeled as zero (class 0).
\end{itemize}

{\em Naive\_Bayes} class implements the well-known naive Bayes
(NB) \cite{cestnik90} classifier. It uses a variation of the Bayes
rule to predict the class for a test instance, assuming that
features are conditionally independent to each other given the
class. NB applies the following rule: \[c_{NB} = \arg \max_{c^{j}
\in C} p(c^{j}) \prod_{i=1}^{n} p(x_{i}|c^{j})\] where $c_{NB}$
denotes the category value predicted by the naive Bayes classifier
for a test instance. The probability for discrete features is
estimated from data using maximum likelihood estimation or
applying the Laplace correction when it is requested by the user.
Unknown values in the test instance are skipped. Although its
simplicity and its conditionally independence assumption among
variables, the literature shows that the NB classifier gives
remarkably high accuracies in many domains \cite{langley94},
specially in medical ones. The {\em Naive\_Bayes} class hosts the
basic facilities to build and manage a naive Bayes supervised
classification model. The following lines are used to describe the
principal methods of this class:

\begin{itemize}
\item public Naive\_Bayes(); It is an empty builder for the class.
\item public Naive\_Bayes(DataBaseCases data, boolean lap); This
class builder fixes the dataset of cases to be used and the
possibility to use the Laplace correction in the learning of the
needed probabilities for the naive Bayes model. \item public void
structuralLearning(); This method learns the naive Bayes
structure. A {\em Bnet} object is built, where each predictive
variable is linked as children node of the class variable. \item
public void parametricLearning(); This method learns the
(conditional) probability distribution tables associated to the
naive Bayes model. \item public static void main(String[] args);
This class is useful when {\em Elvira} is used from the command
line. Given as input a training and test files, this method writes
a file in {\em Elvira} {\em elv} format with the learned naive
Bayes model. The associated confusion matrix is written in the
computer screen.
\end{itemize}

{\em SelectiveNaiveBayes} class implements, with the collaboration
of the class {\em WrapperSelectiveNaiveBayes}, the modification of
the classic naive Bayes approach presented by Langley and Sage
(1994), where not all the predictive variables are included in the
final naive Bayes classification model, which maintains its
conditional independence assumption among the variables (given the
class). A forward hill-climbing search process is performed guided
by a {\em k}-fold cross-validation accuracy estimation, starting
from the empty subset of variables, in the space of variable
subsets. The purpose of this algorithm is to discover the groups
of correlated features which damage the accuracy of a naive Bayes
classifier. The following lines are used to describe the principal
methods of the {\em SelectiveNaiveBayes} class:

\begin{itemize}
\item public SelectiveNaiveBayes(DataBaseCases data, boolean lap);
This class builder fixes the dataset of cases to be used and the
possibility to use the Laplace correction in the learning of the
needed probabilities for the exposed selective naive Bayes model.
\item public abstract void structuralLearning(); This abstract
method is implemented in the {\em WrapperSelectiveNaiveBayes}
subclass, learning the naive Bayes structure using only the subset
of selected predictive variables. \item public void
parametricLearning(); So related with the previous one, this
method learns the (conditional) probability distribution tables
associated to the selective naive Bayes model.
\end{itemize}

{\em WrapperSelectiveNaiveBayes} is a child-class of the {\em
SelectiveNaiveBayes} class. Its principal methods are the
following:

\begin{itemize}
\item public WrapperSelectiveNaiveBayes(); It is an empty builder
for the class. \item public
WrapperSelectiveNaiveBayes(DataBaseCases data, boolean lap, int
k); This class builder calls the class builder with the parameters
{\em data} (dataset of cases), {\em lap} (possibility of using the
Laplace correction in the parametric learning process) and {\em k}
(the number of folds used in the cross-validation process which
guides the search procedure). \item void structuralLearning();
This method learns the selective naive Bayes structure, building a
{\em Bnet} object according to the algorithm proposed by Langley
and Sage (1994), where the search process is performed in the
exposed wrapper form, guided by the estimated accuracy of the
found structure. \item public static void main(String[] args);
This class is useful when {\em Elvira} is used from the command
line. Given as input a training and test files, this method writes
a file in {\em Elvira} {\em elv} format with the learned selective
naive Bayes structure. The associated confusion matrix is also
written in the computer screen.
\end{itemize}


{\em TAN} class implements, with the collaboration of the {\em
CMutInfTAN} class, the classic Tree Augmented Network (TAN)
\cite{friedman97} classifier. This algorithm goes beyond the naive
Bayes classifier as it allow the existence of probabilistic
relationships between the predictive variables: the procedure for
learning these relationships is based on a method reported by Chow
and Liu \cite{chow68}. The authors describe a procedure for
constructing an optimal dependency tree structure in the sense
that among all possible trees, it learns the probabilistic tree
structure that best approximates the data for predictive
variables. The authors use the Kullback-Leibler cross-entropy
measure as a distance criterion between the probability
distribution of the database and the probability distribution
induced by the tree-like structure. Then, the class variable is
linked as parent to every predictive variable. The following lines
are used to describe the principal methods of the {\em TAN} class:

\begin{itemize}
\item public TAN(); It is an empty builder for the class. \item
public TAN(DataBaseCases data, boolean lap); This class builder
fixes the dataset of cases to be used and the possibility to use
the Laplace correction in the learning of the needed probabilities
for the tree augmented network model. \item protected boolean
makesCycle(Graph tree, Node head, Node tail); This method returns
true when the inclusion of the arc {\em head} $\rightarrow$ {\em
tail} creates a cycle in the tree structure. \item public abstract
void structuralLearning(); This abstract method is implemented in
the {\em CMutInfTAN} subclass, learning the tree augmented network
structure. \item public void parametricLearning(); So related with
the previous one, this method learns the (conditional) probability
distribution tables associated to the tree augmented network
model.
\end{itemize}

{\em CMutInfTAN} is a child-class of the {\em TAN} class. Its
principal methods are the following:

\begin{itemize}
\item public CMutInfTAN(); It is an empty builder for the class.
\item public CMutInfTAN(DataBaseCases data, boolean lap); This
class builder calls the {\em TAN} builder with the parameters {\em
data} (dataset of cases) and {\em lap} (possibility of using the
Laplace correction in the parametric learning process). \item void
structuralLearning(); This method learns the tree augmented
network structure, building a {\em Bnet} object according to the
algorithm proposed by Friedman et al. (1997). \item public static
void main(String[] args); This class is useful when {\em Elvira}
is used from the command line. Given as input a training and test
files, this method writes a file in the {\em Elvira's} {\em elv}
format with the learned tree augmented network model according to
the work of Friedman et al. (1997). The associated confusion
matrix is also written in the computer screen.
\end{itemize}


{\em SemiNaiveBayes} class implements, with the collaboration of
the class {\em WrapperSemiNaiveBayes}, the classication approach
presented by Pazzani (1995), called Semi naive Bayes. This
classifier tries to discover in a wrapper way, the dependencies
among predictive features, grouping highly correlated features in
a joint variable (node), which encodes the cartesian product of
dependent (grouped) variables. The classifier also allows the
exclusion of irrelevant predictive variables from the model. A
greedy search process, guided by a {\em k}-fold cross-validation
accuracy estimation, is conducted in the space of allowed
structures, considering the following two operators at each step
of the search:
\begin{enumerate}
\item Add a variable not used by the current classifier as a new
variable class conditionally independent of all other variables
used in the classifier. \item Join a variable not used by the
current classifier with a variable currently used by the
classifier. \end{enumerate}

{\em Elvira's} implementation gathers the forward greedy search
approach of Pazzani's (1995) proposal. The following lines are
used to describe the principal methods of the {\em SemiNaiveBayes}
class:

\begin{itemize}
\item public SemiNaiveBayes(DataBaseCases data, boolean lap); This
class builder fixes the dataset of cases to be used and the
possibility to use the Laplace correction in the learning of the
needed probabilities for the exposed semi naive Bayes model. \item
public abstract void structuralLearning(); This abstract method is
implemented in the {\em WrapperSemiNaiveBayes} subclass, learning
the exposed semi naive Bayes structure. \item public void
parametricLearning(); So related with the previous one, this
method learns the (conditional) probability distribution tables
associated to the semi naive Bayes model. \item public int
assignClass(double[] caseTest); In order to deal with the
join-nodes formed by cartesian products of original variables,
this method is redefined in the {\em SemiNaiveBayes} class to
predict, according to a classifier previously learned, the most
likely class label for a given unlabeled instance.
\end{itemize}

{\em WrapperSemiNaiveBayes} is a child-class of the {\em
SemiNaiveBayes} class. Its principal methods are the following:

\begin{itemize}
\item public WrapperSemiNaiveBayes(); It is an empty builder for
the class. \item public WrapperSemiNaiveBayes(DataBaseCases data,
boolean lap, int k); This class builder calls the class builder
with the parameters {\em data} (dataset of cases), {\em lap}
(possibility of using the Laplace correction in the parametric
learning process) and {\em k} (the number of folds used in the
cross-validation process which guides the search procedure). \item
void structuralLearning(); This method learns the semi naive Bayes
structure, building a {\em Bnet} object according to the algorithm
proposed by Pazzani (1995), where the search process is performed
in the exposed wrapper form, guided by the estimated accuracy of
the found structure. \item public static void main(String[] args);
This class is useful when {\em Elvira} is used from the command
line. Given as input a training and test files, this method writes
a file in {\em Elvira} {\em elv} format with the learned semi
naive Bayes structure. The associated confusion matrix is written
in the computer screen.
\end{itemize}

{\em KDB} class implements, with the collaboration of the {\em
CMutInfKDB} class, the {\em k}-dependence classifier (KDB)
proposed by Sahami (1996). The KDB structure can be seen as a
naive Bayes structure which allows each predictive variable to
have a maximum of {\em k} predictor variables as parents, apart
from the class label, which is linked as parent to every variable:
thus, it allows the construction of classifiers at arbitrary
values for the maximum number of dependencies between variables.
The algorithm proposed by Sahami (1996) uses the class conditional
mutual information mesaure between pairs of predictive variables
and the mutual information between the class label and a single
predictive variable to lead the construction of the classifier.
The following lines are used to describe the principal methods of
the {\em KDB} class:

\begin{itemize}
\item public KDB(int k); It is an empty builder for the class,
where {\em k} denotes the exposed maximum number of parents for
each predictive variable. \item public KDB(DataBaseCases data,
boolean lap, int k); This class builder fixes the dataset of cases
to be used, the possibility to use the Laplace correction in the
learning of the needed probabilities for the {\em k}-dependence
classifier model, and the maximum number of parents for each
predictive variable.  \item public abstract void
structuralLearning(); This abstract method is implemented in the
{\em CMutInfKDB} subclass, learning the {\em k}-dependence
classifier structure. \item public void parametricLearning(); So
related with the previous one, this method learns the
(conditional) probability distribution tables associated to the
{\em k}-dependence classifier model.
\end{itemize}

{\em CMutInfKDB} is a child-class of the {\em KDB} class. Its
principal methods are the following:

\begin{itemize}
\item public CMutInfKDB(); It is an empty builder for the class.
\item public CMutInfKDB(DataBaseCases data, boolean lap, int k);
This class builder calls the {\em TAN} builder with the parameters
{\em data} (dataset of cases), {\em lap} (possibility of using the
Laplace correction in the parametric learning process) and {\em k}
(maximum number of parents for each predictive variable). \item
void structuralLearning(); This method learns the {\em
k}-dependence classifier structure, building a {\em Bnet} object
according to the algorithm proposed by Sahami (1996). \item public
static void main(String[] args); This class is useful when {\em
Elvira} is used from the command line. Given as input a training
and test files, this method writes a file in {\em Elvira} {\em
elv} format with the learned {\em k}-dependence classifier model
according to the work of Sahami (1996). The associated confusion
matrix is written in the computer screen.
\end{itemize}


%- Procedimientos 'te�ricos' para el problema.
%
%    Algoritmo implementado en Elvira.
%
%    Psudoc�digo del algoritmo.
%
%    Referencias bibliogr�ficas.



%- Estructura de clases Principales para resolver el problema.



%-Uso del algoritmo.
%
%    Uso en l�nea de comandos como API
%
%    Uso desde el interface gr�fico
