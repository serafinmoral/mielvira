\section{Introduction}

Classification is one of the most important subfields of {\em
machine learning} \cite{mitchell97}. As another machine learning
task, the goal of a classification process is to build, starting
from a set of examples (the training experience), a general rule
(or classification model).

An example describes the basic entity of the reality to be
studied, a part of a known experience, such as a medical patient,
a DNA microarray extracted from a tissue, a credit request or an
e-mail received in our inbox. A variable (or feature) describes a
specific characteristic of an example, such as the output of a
medical test, the expression level of a specific gene, the age of
a credit requester or the existence/nonexistence of a specific
word in an e-mail.

In {\em supervised classification} every example has a special
feature, known as the class label, which describes the value of
the phenomenom of interest to be studied, {\em i.e.}, the
diagnostic of the medical patient, the `tumor' or `healthy' nature
of the studied tissue, the possible fraud of the credit requester
or the `spam' nature of the received e-mail. It can be considered
that the class label of each example is provided by a supervisor
or teacher. Starting from the dataset of labelled examples, the
class label supervises the induction process of the classifier,
which is used in a second step to predict, the most accurately as
possible, the class label for further unlabelled examples. Thus, a
bet is performed, classifying the unlabelled example with the
label that the classifier predicts for it.

In {\em data clustering} the task is characterized by the
inexistence of this class label in the domain at hand: this makes
to also refer to this paradigm as {\em unsupervised
classification}. Clustering assumes that data is characterized by
an underlying group-structure, and its objective is to provide a
`good' description of this group-structure, when this group
membership is unknown.

{\em Elvira} includes several facilities for the building and
manipulation processes of several well-known supervised and
unsupervised classification techniques. These facilities can be
easily acceded by the graphical interface or by on-line command
instantiation of the specific {\em Elvira} classes. The following
sections are going to be devoted to the explanation of these
supervised and unsupervised tools in the {\em Elvira} software.
