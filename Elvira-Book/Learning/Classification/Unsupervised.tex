\section{Unsupervised Classification}

{\em Elvira} allows the possibility of building a naive Bayes
model for unsupervised classification (or clustering), where the
class variable is latent or hidden. Elvira offers the possibility
for learning probabilistic parameters of the naive Bayes
unsupervised model by two different methods: a single learning
process of the well-known EM method \cite{dempster77,mclachlan97}
and a multi-start process of the EM method (i.e., several runs of
the EM method starting from different initial conditions).

The facilities for using the proposed learning methods are easily
accessible from the graphical interface of {\em Elvira}.

{\em Elvira} has two Java classes which implement these
classifiers and facilities: {\em NBayesMLEM} and {\em
NBayesMLEMStart}. These classes are located in the {\em
elvira.learning.classification.unsupervised.discrete} package.

The {\em NBayesMLEM} class implements the EM algorithm
\cite{dempster77,mclachlan97} for learning the parameters of the
explained unsupervised naive Bayes model. The EM is an iterative
algorithm which starts from an initial configuration of the naive
Bayes parameters (this initial configuration is set at random in
{\em Elvira}); the maximum likelihood parameters are approximated
by the following two steps in each iteration:
\begin{itemize}
\item E step or Expectation: given the current estimation of the
unsupervised naive Bayes parameters, the expected $N_{ijk}$ values
are calculated ($N_{ijk}$ denotes the number of cases in the
dataset in which the $i$-th predictive variable shows its $k$-th
value and its parents their $j$-th value-configuration. It is
obvious that these parameters are adapted to the naive Bayes
structure); \item M step or Maximization: given the expected
$N_{ijk}$ calculated in the previous step, this step re-estimates
the parameters for the naive Bayes model to be the maximum
likelihood.
\end{itemize}

The EM is a greedy algorithm. It converges to a local maximum
which is not necessarily the same as the global maximum, that is,
the maximum likelihood parameter configuration for the
unsupervised naive Bayes model.

The {\em NBayesMLEM} class has the following two methods:

\begin{itemize}
\item public NBayesMLEM(DataBaseCases cases, int
numberofClusters); This method adds the latent class variable,
which is assumed to have {\em numberOfClusters} different values,
and it builds the unsupervised naive Bayes model. The class
variable must not be described in the dataset, where no missing
values are allowed. \item public double learning(boolean
laplaceCorrection); This method implements the exposed EM
iterative algorithm to learn the parameters for the unsupervised
naive Bayes model. When the {\em laplaceCorrection} parameter is
set to true, the Laplace correction is used to calculate the
maximum likelihood parameters.
\end{itemize}

The {\em NBayesMLEMStart} class also learns the parameters of the
unsupervised naive Bayes model by means of the EM algorithm.
However, this class is used to perform several runs of the EM
algorithm, being the initial parameters configuration for each run
chosen at random. Thus we have a better chance of obtaining the
global maximum likelihood parameter configuration for the
unsupervised naive Bayes model. The result of this EM multi-start
process is the best model from the whole set of runs, that is, the
model with the largest log-likelihood given the dataset of cases.

The {\em NBayesMLEMStart} class has the following two methods:
\begin{itemize}
\item public NBayesMLEMStart(DataBaseCases cases, int
numberofClusters); This method adds the latent class variable,
which is assumed to have {\em numberOfClusters} different values,
and it builds the unsupervised naive Bayes model. The class
variable must not be described in the dataset, where no missing
values are allowed. The first method of the {\em NBayesMLEM} class
is identical to this one. \item public double learning(boolean
laplaceCorrection, int N); This method implements the exposed
multi-start process of the EM algorithm to learn the unsupervised
naive Bayes model. When the {\em laplaceCorrection} parameter is
set to true, the Laplace correction is used to calculate the
maximum likelihood parameters. The {\em N} parameter indicates the
number of runs to be performed in the multi-start process.
\end{itemize}
