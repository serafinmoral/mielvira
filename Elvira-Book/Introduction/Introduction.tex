\section{Introduction}

Elvira is a software tool written in Java for construction and use of Probabilistic Graphical Models.
It has been built with the support of the Spanish
{\em Ministerio de Ciencia y Tecnolog\'{\i}a} through two
research projects: Elvira (TIC97-1135-C04, 1997-2000) and Elvira~II
(TIC2001-2973-C05, 2001-2004), with the participation of five partners:
University  of Almer�a, University of the Basque Country, University of Castilla La Mancha, University of
Granada, and the  Spanish University for Distance Education.

The main aim of Elvira was to be used for research. The idea was to have a common framework in which
the implementation of a new algorithm for some particular task was simple using the available data
structures and, at the same time making easier  the comparison  experiments providing an important collection
 of existing algorithms
for the same task and  standard procedures to evaluate their performance. Nevertheless, Elvira
can be also used to build real applications for concrete problems as it has a friendly interface
to use graphical models.

As a consequence, the  main characteristics of the software are:

\begin{itemize}


\item Elvira is multipurpose. It is for Bayesian networks (learning, inference, use),
influence diagrams, and classification (both supervised and non-supervised).

\item We have not selected an unique method or algorithm for each particular task.
There are several ones available and in some cases, as in approximated evaluation
on Bayesian networks, the number of possibilities is quite high.

\item The programs are  changing dynamically. There are  several programmers
working on the tool in parallel who incorporate modifications in a continuous way.

\item There is not a plan to finish the project in the near future or even to produce
an stable version of it. However, we also think that an stable version would
be necessary in the long term.

\end{itemize}

Some of these characteristics can be seen as defects, but we feel that
they provide the necessary flexibility for serving the research interests
of the different participating groups.

From the beginning we felt that our effort could also be useful to other
research groups over the world and Elvira can be downloaded from
its home page at: http://leo.ugr.es/~elvira. But in this moment,
the utility of the software can limited by the lack of a general
reference of all the implemented methods and procedures.
Only  \cite{elvira02} is a summary of the state of Elvira in 2002.
But now, Elvira programs are so large that it is even difficult
for the participants to have an idea of everything included
in it. It is for that reason that we have thought in writing this
book which could serve as a guide for researchers and
general users of Elvira.

Each one of the chapters is devoted to a concrete aspect
of Elvira. Three levels of description are considered:

\begin{enumerate}

\item {\em Abstract level.-} The algorithms are described in a precise
but, at the same time, brief way. References are given to more
detailed accounts of them.

\item {\em User level.-} It is explained how to use the algorithms and facilities
from the graphical interface.

\item {\em Programmer level.-} The main procedures with their
arguments are detailed. There are a lot of functions which only
can be used at this level, being not available from the graphical
interface.


\end{enumerate}


\section{Downloading and Installing Elvira}

Elvira can be downloaded from its main web page \cite{elviraweb}.
It is  a .tar.gz file. It can be decompressed  in Unix or Linux by using:
{\tt tar zxvf bayelvira2.tar.gz}.
In Windows 95, 98, 2000, XP, it can be decompressed by using the Winzip program.

Detailed guidelines to use Elvira can be found in \cite{elvirainstall}.
The easiest way to run Elvira in Windows is by double-clicking on the icon
Elvira.jar in the installation directory.

In linux, you must open a command shell on the directory in which you installed
 Elvira and enter the command {\tt java -jar Elvira.jar}.
  Elvira will automatically detect the language of your operating system.
   Alternatively, you can write {\tt java -jar Elvira.jar -l ae} to
see the interface in American English or {\tt java -jar Elvira.jar -l sp} to see it in Spanish.



\section{Main Characteristics of Elvira}

There are other programs to work with probabilistic graphical models \cite{hugin,genie,analytica}.
See \cite{http://www.cs.ubc.ca/~murphyk/Bayes/bnsoft.html} for a quite complete list of Bayesian
networks software. Elvira has some unique characteristics which make it very appropriate for
some task, mainly research oriented work, but without discarding to be the basis for building
real applications. Among them, we want to highlight the following ones:

\begin{itemize}

\item The definition of a very flexible format to store networks, databases, sets of observations,
and experiments results. This format is very intuitive and can be edited with any text editor.
It is described in detail in Chapter \ref{Ch:format}.

\item A great variety of data structures. Apart from the array representation of conditional
probability distributions, it also provides a representation by means of trees. This structure
can improve the efficiency when there are asymmetrical independence relationships in the problem
and are not implemented in other software tools.
Elvira gives the possibility of using several types of canonical functions as noisy or-gates to simplify the specification of conditional
probabilities when the number of parents is high. These aspects are described in Chapter \cite{Ch:data}.

\item Elvira has a very important collection of algorithms for exact and approximate inference,
both to compute the 'a posteriori' probability distribution and for partial abductive inference.
Partial abduction is rarely found in any of existing tools.  These algorithms are described
in Chapters \ref{Ch:exact} (exact inference) and \ref{Ch:approx} (approximate inference).

\item Elvira is multipurpose. It includes Bayesian networks, but also models for supervised and unsupervised classification
(see Chapter \ref{Ch:class}).  Also graphical models for decision problems (influence diagrams) can be built and used
in Elvira, with bout exact and approximate algorithms (see Chapter \red{Ch:influence}).

\item Elvira has a really collection of learning algorithms: based on independence tests and based on search + score.
Different variations of each one of them are considered (based on different statistical tests, methods of search, or
score functions). A detailed description of them can be found in Chapter \ref{Ch:learning}.

\item The graphical interface has some special functions to explain the nature  of the represented knowledge (different colors and widths for the links)
and the reasoning, allowing to follow the reasons for changes in probability when entering evidence (see Chapter \ref{Ch:explain}).

\item As Elvira has been designed as a research tool, it has general procedures to carry out experiments. So, it has
classical validation procedures for supervised classification as K-fold cross validation, also the computation of
measures of discrepancy  within an original true model and the original one (differences in the structure or the
computation of Kullback-Leibler distance between to Bayesian networks).



\end{itemize}

\section{Elvira Structure: Graphical Interface}

Elvira has three main 


\section{Classes Estructure}



\section{Future Evolution of Elvira}
